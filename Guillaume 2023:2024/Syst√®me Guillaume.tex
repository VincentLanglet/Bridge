\documentclass[a4paper, oneside, 11pt]{report}
\usepackage{Bridge}

\begin{document}
\Chap{Ouverture de 1SA}
	\Section{Développements}
		\SSection{Réponses}
		\begin{tabular}{cccc|l}
		1SA & - & 2\trefle && Stayman\\
		&& 2\carreau/2\coeur && Texas \coeur/\pique\ (Rectification au palier de 3 = 4 atouts maximum)\\
		&& 2\pique && Texas \trefle\ (Rectification à 2SA = 3 atouts dont 1 honneur ou 4 atouts)\\
		&& 2SA && Limite\\
		&& 3\trefle && Texas \carreau\\
		&& 3\carreau && Chelemisant naturel\\
		&& 3\coeur/3\pique && Chelemisant naturel\\
		&& 3SA/4\coeur/4\pique && Pour jouer\\
		&& 4\trefle && Bicolore majeur au moins 6/5 (4\carreau\ sera un relais pour la 6$^{ème}$)\\
		&& 4\carreau && Bicolore majeur 5/5\\
		\end{tabular}\\\\

		Après un Texas Majeur,  2SA est propositionnel.\\
		Après un Texas Mineur,  on nomme les singletons majeurs naturellement.\\
		Après 1SA - 2\trefle\ - 2SA,  on nomme la couleur en sous-texas (ou naturellement pour chelemiser).\\

		\underline{Cas de l'intervention par 1SA :} Minou - Matou
	
		En cas de contre punitif, on joue naturel :\\
		\begin{tabular}{cccc|l}
		1X & 1SA & X & Passe & 4333 (le partenaire dégage avec une 5$^{ème}$)\\
		&&& XX & 4432 ou 4441 (Puis Baron)\\
		&&& 2\trefle/2\carreau/2\coeur/2\pique & Naturel faible\\
		&&& 3\trefle/3\carreau/3\coeur/3\pique & Naturel limite\\
		\end{tabular}\\\\

	\Section{Réaction face aux interventions}
		\SSection{Contre du Stayman/Texas}
		Si le Texas est contré, on fait jouer de l'autre main :
		\begin{itemize}
		\item Passe = Non fitté
		\item XX = Fitté\\
		\end{itemize}
		
		Si le 2\trefle\ Stayman est contré (Idem si intervention de 2\carreau) :\\
		\begin{tabular}{cccc|l}
		1SA & - & 2\trefle & X &\\
		Passe &&&& Les autres mains (Typiquement des 4333)\\
		XX &&&& Pour jouer\\
		2\carreau &&&& Naturel\\
		2\coeur/\pique &&&& 4\pique/\coeur\ (Inversé pour mettre le X à l'entame)\\
		2SA &&&& Un arrêt solide\\
		3\trefle &&&& Les deux majeurs (Pas de Texas \& Le palier de 3 est chelemisant)\\
		\end{tabular}\\\\

		\begin{tabular}{cccc|l}
		1SA & - & 2\trefle & X &\\
		2\pique & - & 2SA && Limite avec ou sans l'arrêt \trefle\\
		&& 3\trefle && Demande d'arrêt \trefle\ pour 3SA OU Fitté \coeur\ chelemisant\\
		&& 3\coeur && Fitté \coeur\ limite\\
		\end{tabular}\\\\
		
		\begin{tabular}{cccc|l}
		1SA & - & 2\trefle & X &\\
		3\trefle & - & 3\carreau && Propositionnel (Sur 3\coeur\ minimum, 3\pique\ est pour jouer)\\
		&& 3\coeur/3\pique && Chelemisant\\
		&& 4\coeur/4\pique && Pour jouer\\
		\end{tabular}\\\\
		
\newpage
		\SSection{Intervention sur le Stayman/Texas}
		\begin{itemize}
		\item Après un Stayman,  le contre de l'ouvreur est punitif,  celui du répondant est d'appel
		\item Après un Texas,  le contre de l'ouvreur est punitif,  celui du répondant est d'appel\\
		\end{itemize}

		\SSection{Rubensohl}
		\begin{tabular}{cccc|l}
		1SA & 2\coeur\ (Naturel) & X && Appel\\
		&& 2\pique\ && Naturel\\
		&& 2SA/3\trefle && Texas \trefle/\carreau\\
		&& 3\carreau && Texas impossible = Stayman\\
		&& 3\coeur && Texas \pique\ FM (Sans l'arrêt coeur)\\
		&& 3\pique && 5\pique\ (Avec l'arrêt coeur)\\
		\end{tabular}\\
		Même idée pour l'intervention par 2\pique\\

		\begin{tabular}{cccc|l}
		1SA & 3\trefle\ (Naturel) & X && Stayman\\
		&& 3\carreau/3\coeur && Texas \coeur/\pique\\
		&& 3\pique && Texas \carreau\\
		\end{tabular}\\\\

		\begin{tabular}{cccc|l}
		1SA & 3\carreau\ (Naturel) & X && Stayman\\
		&& 3\coeur && Texas \pique\\
		&& 3\pique && Texas \coeur\\
		\end{tabular}\\\\

		\SSection{Défense contre le Landy}
		\begin{tabular}{cccc|l}
		1SA & 2\trefle\ (Landy) & X && Punitif dans au moins une majeure OU 8H+ sans arrêt\\
		&& 2\coeur/2\pique\ && Singleton \coeur/\pique\ avec un 54 mineur\\
		&& 2SA/3\trefle && Texas \trefle/\carreau\\
		&& 3\carreau && Une majeure cinquième\\
		&& 3\coeur/3\pique && Chicane \coeur/\pique\ avec un 55 mineur\\
		\end{tabular}\\\\

		\SSection{Défense contre le Multi}
		\begin{tabular}{cccc|l}
		1SA & 2\carreau\ (Multi) & X && Appel généralisé\\
		&& 2\coeur/2\pique\ && Naturel non forcing\\
		&& 2SA/3\trefle && Texas \trefle/\carreau\\
		&& 3\carreau && Texas \coeur\ FM avec ou sans l'arrêt \pique\\
		&& 3\coeur && Texas \pique\ FM sans l'arrêt \coeur\\
		&& 3\pique && 5\pique\ avec l'arrêt \coeur\\
		\end{tabular}\\\\

		\begin{itemize}
		\item 1SA 2\carreau\ X Passe Passe 2\coeur\ X  = Punitif
		\item 1SA 2\carreau\ X Passe Passe 2\coeur\ Passe  = Forcing
		\item 1SA 2\carreau\ X Passe Passe 2\coeur\ Passe  Passe X = Punitif
		\item 1SA 2\carreau\ Passe 2\coeur\ X = Appel
		\item 1SA 2\carreau\ X 2\coeur\ X = Punitif\\
		\end{itemize}

		\SSection{Défense contre le X}
		\begin{tabular}{cccc|l}
		1SA & X & Passe && Forcing\\
		&& XX && 8H+\\
                  && Autre && On ignore le contre\\
		\end{tabular}\\\\

		\begin{tabular}{cccc|l}
		1SA & X & - & - & \\
                  XX &&&& Obligatoire sans couleur 5$^{ème}$ (Puis baron)\\
		2\trefle/2\carreau/2\coeur/2\pique &&&& Naturel dans 5 cartes\\
		\end{tabular}\\

\Chap{Ouverture de 2\trefle/2\carreau/2\coeur/2\pique/2SA et 3SA}
	\Section{2\trefle\ Bivalent}
	2\trefle\ regroupe les mains régulières 23+, les deux forts en majeur et les mains FM.\\
	
	\begin{tabular}{cccc|l}
	2\trefle & - & 2\carreau && Relais FM\\
	&& 2\coeur && Pour jouer en face d'un 2\coeur\ fort\\
	&& 2\pique && Pour jouer en face d'un 2\pique\ fort\\
         && 2SA && L'enchère de 2\carreau\ avec un gros bicolore mineur\\
	&& 3\trefle && Jeu faible, au moins 5-4 en majeur (Stayman)\\
	&& 3\carreau && Jeu faible, 4-4 en majeur (Stayman)\\
	\end{tabular}\\\\
	
	\begin{tabular}{cccc|l}
	2\trefle & - & 3\trefle & - &\\
	3\carreau &&&& Demande d'une majeur cinquième (réponse en chassé croisé)\\
	3\coeur/3\pique/4\trefle/4\carreau &&&& Naturel chelemisant\\
	3SA &&&& Pour jouer\\
	4\coeur/4\pique/5\trefle/5\carreau &&&& Pour jouer\\	
	\end{tabular}\\
	Même principe pour la séquence 2\trefle\ - 3\carreau\\

	En cas d'intervention X (ou XX) négatif et Passe encourageant\\	

	\Section{2\carreau\ Multi}
	Version faible : 2\carreau\ est un unicolore majeur faible indéterminé\\

	\begin{tabular}{cccc|l}
	2\carreau & - & 2\coeur/2\pique/3\coeur/3\pique && Passe ou Corrige\\
	&& 2SA && Relais\\
	&& 3\trefle && Naturel ; Pour jouer\\
	&& 3\carreau && Forcing avec l'autre majeure présumée\\
	&& 4\trefle && Nomme ta majeur en Texas\\
	&& 4\carreau && Nomme ta majeur\\
	&& 4\coeur/4\pique && Naturel\\
	\end{tabular}\\\\

	\begin{tabular}{cccc|l}
	2\carreau & - & 2SA & - &\\
	3\trefle/3\carreau &&&& Minimum \coeur/\pique\ (3\carreau/\coeur\ pour vérifier qu'il s'agit d'un vrai barrage)\\
	3\coeur/3\pique &&&& Maximum \pique/\coeur\\
	\end{tabular}\\\\
	
	\begin{tabular}{cccc|l}
	2\carreau & - & 3\carreau & - &\\
	3\coeur &&&& Unicolore \coeur\ sans 3\pique\\
	3\pique &&&& Unicolore \pique\ sans 3\coeur\\
	4\trefle &&&& Unicolore \coeur\ avec 3\pique\\
	4\carreau &&&& Unicolore \pique\ avec 3\coeur\\
	\end{tabular}\\\\

	\begin{tabular}{cccc|l}
	2\carreau & X & Passe && Du carreau ; Accepte 2\carreau\ X\\
	&& XX && Pour jouer 2\coeur\ ou 2\pique\\
	&& 2\coeur && Pour jouer 2\coeur\ ou 3\pique\\
	&& 2\pique/3\coeur/3\pique && Passe ou Corrige\\
	&& 4\trefle && Nomme ta majeure en Texas\\
	&& 4\carreau && Nomme ta majeure\\
	&& 4\coeur/4\pique && Naturel\\
	\end{tabular}\\\\
	En cas d'intervention naturelle majeure, le contre est passe ou corrige\\	
	En cas d'intervention naturelle par 3\trefle, les enchères 4\trefle\ et 4\carreau\ gardent leur sens\\	
	En cas d'intervention naturelle par 3\carreau, 4\trefle\ est non forcing et 4\carreau\ demande la majeur\\

\newpage
	\Section{2\coeur\ \& 2\pique\ Muiderberg}
	2\coeur/\pique\ est un bicolore faible au moins 5\coeur/\pique\ et 4\trefle/4\carreau\\

	\begin{tabular}{cccc|l}
	2\pique & - & 2SA && Relais descriptif\\
	&& 3\trefle/4\trefle/5\trefle && Passe ou Corrige\\
	&& 3\coeur && 6\coeur\ forcing\\
	&& 3\pique/4\pique && Barrage\\
	&& 3SA && Pour jouer\\
	\end{tabular}\\\\
	
	Même principe sur 2\coeur\ avec les enchères:
	\begin{tabular}{cccc|l}
	2\coeur & - & 2\pique && Passe (doubleton) ou Corrige\\
	&& 3\pique && 6\pique\ forcing\\
	\end{tabular}\\\\

	\begin{tabular}{cccc|l}
	2\coeur/\pique & - & 2SA & - &\\
	3\trefle/3\carreau &&&& 5\coeur/\pique\ et 4\trefle/4\carreau\ Minimum\\
	3\coeur/3\pique &&&& 5\coeur/\pique\ et 4\trefle/4\carreau\ Maximum\\
	4\trefle/4\carreau &&&& 5\coeur/\pique\ et 5\trefle/5\carreau\ Maximum\\
	\end{tabular}\\\\
	
	Sur les réponses de 3\trefle/3\carreau/3\coeur/3\pique:
	\begin{itemize}
	\item 3\coeur/\pique\ est pour les jouer
	\item 4\trefle/\carreau\ est chelemisant dans la mineure
	\item Le reste est chelemisant dans la majeure\\
	\end{itemize}
	
	Après X ou sur une ouverture en $3^{ème}$, 2SA devient un relais pour la mineur et 3\trefle/\carreau\ sont naturels\\

	\Section{2SA}
	2SA annonce 20-22HL. Les réponses sont standards. \\
La rectification du Texas est fittée par 4 et obligatoire par paire.\\

	\Section{3SA Gambling}
	\begin{tabular}{cccc|l}
	3SA & - & 4\trefle && Passe/Corrige\\
	&& 4\carreau && Demande de singleton\\
	&& 4\coeur/4\pique && Pour jouer\\
	&& 5\trefle/6\trefle && Passe ou corrige\\
	\end{tabular}\\\\
	
	\begin{tabular}{cccc|l}
	3SA & - & 4\carreau & - &\\
	4\coeur &&&& Singleton \coeur\\
	4\pique &&&& Singleton \pique\\
	4SA &&&& 7222\\
	5\trefle &&&& Singleton \carreau\ donc avec les \trefle\\
	5\carreau &&&& Singleton \trefle\ donc avec les \carreau\\
	\end{tabular}\\\\

	\Section{En 4$^{ème}$}
	\begin{tabular}{cccc|l}
	- & - & - & 2\trefle\ & Bivalent\\
	&&& 2\carreau\ & 10-13 ; 6 cartes \carreau\\
	&&& 2\coeur/2\pique & 10-13 ; 6 cartes \coeur/\pique\\
	&&& 2SA & 20-22 HL régulier\\
	\end{tabular}\\\\
	
\Chap{Ouverture majeure}
	\Section{Fit majeur (Sans passe d'entrée)}
		\SSection{Réponses}
		\begin{tabular}{cccc|l}
		1\coeur & - & 1SA && Peut cacher un fit faible (ou une main limite régulière)\\
		&& 2\coeur && 6-10 ; Fitté par 3\\
		&& 2\pique && 6\pique\ faible (4-6)\\
		&& 2SA && Fitté limite par 3 ou 4 OU Fitté par 3 Régulier 12-14  \\
		&& 3\trefle/\carreau && 6 cartes limite\\
		&& 3\coeur && 7-9 ; Fitté par 4 (Mixed raise)\\
		&& 3\pique/4\trefle/4\carreau && Splinter\\
		&& 3SA && Fitté par 4 Régulier 12-14\\
		&& 4\coeur && Barrage\\
		\end{tabular}\\

		Même principe sur 1\pique, sauf l'enchère de 3\coeur, qui est une main limite, sans le fit \pique,  avec 6\coeur\\

		\SSection{En cas d'intervention}
		\begin{tabular}{cccc|l}
		1\coeur & X & XX && Jamais fitté\\
		&& 2\carreau && Fit constructif\\
		&& 2\coeur && Fit barrage\\
		&& 2\pique && Cue-bid imaginaire = Fit de manche par 4\\
		&& 2SA && Fitté limite par 3 ou 4 OU Fitté par 3 de manche\\
		&& 3\coeur && Barrage\\
		&& 3SA && Naturel\\
		&& 4\coeur && Barrage\\
		\end{tabular}\\\\

		\begin{tabular}{cccc|l}
		1\coeur & 2\trefle & X && Jamais fitté\\
		&& 2\coeur && Classique\\
		&& 2SA && Fitté limite par 3 ou 4 OU Fitté par 3 de manche\\
		&& 3\trefle && Cue-bid = Fit de manche par 4\\
		&& 3\coeur && Barrage\\
		&& 3SA && Naturel\\
		&& 4\coeur && Barrage\\
		\end{tabular}\\\\

		Si X ou intervention, les changements de couleur à saut sont fittés\\

		\SSection{Fit majeur (Après passe d'entrée)}
		\begin{tabular}{cccc|l}
		& (-) & - & - &\\
		1\coeur/\pique & - & 2\trefle & - &\\
		2\carreau &&&& Demande de description\\
		2\coeur/\pique &&&& Pour jouer\\
		4\coeur/\pique &&&& Pour jouer\\
		\end{tabular}\\
		Après une ouverture d'1\pique, 2\coeur\ promet un jeu sans l'ouverture dans un 5/4\\

		\begin{tabular}{cccc|l}
		& (-) & - & - &\\
		1\coeur/\pique & - & 2\trefle & - &\\
		2\carreau & - & 2\coeur/\pique && Main limite fittée par 3 cartes\\
		&& 2SA && Main limite fitté par 4 cartes avec un singleton\\
		&& 3\coeur/\pique && Main limite fittée par 4 cartes\\
		\end{tabular}\\\\
		
		La demande de singleton se fait à 3\trefle\ on répond les courtes naturelles.\\

\newpage
	\Section{Séquences particulières}
		\SSection{2/1 forcing de manche}
		\begin{tabular}{cccc|l}
		1\coeur & - & 2\carreau & - &\\
		2\pique &&&& Naturel,  ne promet pas plus que l'ouverture\\
		2SA &&&& 17-19,  5332 avec 3\carreau\\
		3\trefle &&&& Naturel 15+\\
		3\carreau &&&& 4\carreau\, ne promet pas plus que l'ouverture\\
		3SA &&&& 17-19,  5332 avec 2\carreau\\
		\end{tabular}\\
		La séquence 1\pique\ 2\coeur\ 3\coeur\ promet toujours 15+ (avec 12-14 on dit 4\coeur)\\

		\begin{tabular}{cccc|l}
		1\pique & - & 2\carreau & - &\\
		2\coeur & - & 2\pique && Chelemisant avec 3\pique\\
		&& 2SA && Relais naturel FM promettant 2\pique\\
		&& 3\pique && Chelemisant avec 4\pique\\
		&& 3SA && Pour jouer (dénie 2\pique)\\
		\end{tabular}\\\\

		\begin{tabular}{cccc|l}
		1\pique & - & 2\carreau & - &\\
		2\pique & - & 2SA && Relais naturel FM promettant 2\pique\ (ne dénie pas 3-4\pique)\\
		&& 3\pique && Chelemisant avec 4\pique\\
		&& 3SA && Pour jouer (dénie 2\pique)\\
		\end{tabular}\\\\

		\begin{tabular}{cccc|l}
		1\pique & - & 2\trefle & - &\\
		2\pique & - & 2SA & - &\\
		3\trefle/3\carreau/3\coeur &&&& 6\pique\ singleton \trefle/\carreau/\coeur\\
		3\pique &&&& 6\pique\ sans singleton (3SA est un relai pour les contrôles)\\
		3SA &&&& 5332 (4\trefle/4\carreau/4\coeur\ est alors un contrôle à l'atout \pique)\\
		\end{tabular}\\\\

		\begin{tabular}{cccc|l}
		1\pique & - & 2\coeur & - &\\
		2\pique & - & 2SA & - &\\
		3\trefle/3\carreau &&&& Naturel 12-14 sans 3\coeur\\
		3\coeur &&&& 6\pique\ singleton \coeur\\
		3\pique &&&& 6\pique\ sans singleton (3SA est un relai pour les contrôles)\\
		3SA &&&& 5332  (4\trefle/4\carreau/4\coeur\ est alors un contrôle à l'atout \pique)\\
		4\trefle/4\carreau &&&& 6\pique\ singleton \trefle/\carreau\\
		4\coeur &&&& 12-14 avec 3\coeur\\
		\end{tabular}\\\\

		\SSection{1\coeur\ 1\pique\ 3SA}
		\begin{tabular}{cccc|l}
		1\coeur & - & 1\pique & - &\\
		3SA &&&& 18-19 avec 5\coeur\ et 4\pique\\\
		4\pique &&&& "Gambling" avec 6\coeur\ et 4\pique\\
		\end{tabular}\\\\

		\SSection{1\pique\ 1SA 2\trefle\ 2\carreau}
		\begin{tabular}{cccc|l}
		1\pique & - & 1SA & - &\\
		2\trefle & - & 2\carreau && 5\coeur\ et un repli à 2\pique\ ou 3\trefle\\
		&& 2\coeur && 6\coeur\\
		\end{tabular}\\\\

		\SSection{Oui-mais}
		Lors d'une séquence chemisante à \coeur, exemple 1\coeur\ 2\carreau\ 2\coeur\ 3\coeur\ :
		\begin{itemize}
		\item 3\pique\ est Oui-mais (3SA sera le contrôle \pique)
		\item 3SA annonce le contrôle \pique\\
		\end{itemize}
		
\Chap{Ouverture mineure}
	\Section{Soutien mineur inversé}
		\SSection{Principe}
		\begin{tabular}{cccc|l}
		1\trefle & - & 2\trefle && Soutien FM (Possiblement 4\coeur/\pique)\\	
		&& 2\carreau && Soutien limite (9-11HL)\\
		&& 3\trefle && Soutien barrage (6-8 HL)\\
		\end{tabular}\\
		Après passe, le SMI n'est plus FM et dénie 4\coeur/\pique\\
	
		\begin{tabular}{cccc|l}
		1\trefle & - & 2\trefle & - &\\
		2\carreau/2\coeur/2\pique &&&& Naturel irrégulier (dès 12 points)\\
		2SA &&&& 12-14 OU 18-19 ; Régulier\\
		3\trefle &&&& Unicolore 6$^{ème}$ (Ou 5$^{ème}$ avec un petit doubleton majeur)\\
		3\carreau/3\coeur/3\pique &&&& Bel unicolore 6$^{ème}$ ; Splinter\\
		3SA &&&& 15-17 ; 4441 Singleton \carreau\\
		4\trefle &&&& Chelemisant \trefle\\
		\end{tabular}\\\\

		Même principe sur l'ouverture d'1\carreau\\

		\SSection{Développement sur 2SA}
		\begin{tabular}{cccc|l}
		1\trefle/\carreau & - & 2\trefle/\carreau & - &\\
		2SA & - & 3\trefle && Relais pour 3\carreau\\
		&& 3\carreau/3\coeur && 4\coeur/4\pique\\
		&& 3\pique && Bicolore 5\trefle/5\carreau\\
		&& 3SA && Pour jouer\\
		&& 4\trefle/\carreau && Chelemisant\\
		\end{tabular}\\\\

		\begin{tabular}{cccc|l}
		1\trefle/\carreau & - & 2\trefle/\carreau & - &\\
		2SA & - & 3\trefle & - &\\
		3\carreau & - & 3\coeur/\pique && Singleton \coeur/\pique\\	
		&& 3SA && Singleton \carreau/\trefle\ (Autre mineur)\\
		&& 4\trefle/\carreau && Singleton \carreau/\trefle\ (Autre mineur) ; Chelemisant\\
		\end{tabular}\\\\

		\begin{tabular}{cccc|l}
		1\trefle/\carreau & - & 2\trefle/\carreau & - &\\
		2SA & - & 3\carreau & - &\\
		3\coeur &&&& 18-19 ; Fitté\\
		3SA &&&& 12-14 ; Non fitté\\
		4\coeur &&&& 12-14 ; Fitté\\
		4SA &&&& 18-19 ; Non fitté\\
		\end{tabular}\\
		De même pour les \pique\\

		\SSection{Développement sur 3SA}
		\begin{tabular}{cccc|l}
		1\trefle/\carreau & - & 2\trefle/\carreau & - &\\
		3SA & - & 4\trefle && Chelemisant \trefle/\carreau\\
		&& 4\carreau/4\coeur && 4\coeur/4\pique\ (Le fit est certain)\\
		&& 4SA && Quantitatif\\
		\end{tabular}\\\\

\newpage
		\SSection{Séquence 1\carreau\ - 2\trefle}
		\begin{tabular}{cccc|l}
		1\carreau & - & 2\trefle & - &\\
		2\carreau &&&& Unicolore 6$^{ème}$ OU Bicolore de première zone\\
		2\coeur/2\pique &&&& Naturel irrégulier (dès 15 points) \\
		2SA &&&& 12-14 OU 18-19 ; Régulier\\
		3\trefle &&&& Naturel irrégulier (dès 15 points)\\
		3SA &&&& 15-17 ; 4441 Singleton \trefle\ (Puis tout Texas)\\
		\end{tabular}\\\\

		\begin{tabular}{cccc|l}
		1\carreau & - & 2\trefle & - &\\
		2\carreau & - & 2\coeur/2\pique && 4\coeur/\pique\ OU Un arrêt pour 3SA\\
		&& 2SA && Relais FM (demande de courte)\\
		&& 3\trefle && Naturel FM\\
		&& 3\carreau && Chelemisant \carreau\\
		&& 3\coeur/3\pique && Fit \carreau\ ; Splinter \coeur/\pique\\
		\end{tabular}\\\\

		\begin{tabular}{cccc|l}
		1\carreau & - & 2\trefle & - &\\
		2\carreau & - & 2\coeur & - &\\
		2SA &&&& L'arrêt \pique\ sans fit \coeur\\
		3\coeur &&&& Le fit \coeur\ sans l'arrêt \pique\\
		3SA&&&& Le fit \coeur\ et l'arrêt \pique\\
		\end{tabular}\\\\

		\begin{tabular}{cccc|l}
		1\carreau & - & 2\trefle & - &\\
		2SA & - & 3\trefle && Relais pour 3\carreau\ (Puis même développement que le SMI)\\
		&& 3\carreau/3\coeur && 4\coeur/4\pique\ (Puis même développement que le SMI)\\
		&& 3\pique && Proposition de chelem à \trefle\ (3SA décourageant)\\
		&& 3SA && Pour jouer\\
		&& 4\trefle/4\carreau && Naturel chelemisant\\
		\end{tabular}\\\\

\newpage
	\Section{Autres conventions}
		\SSection{Check back stayman}
		Le check back stayman se fait dans l'autre mineure.

		Avec les deux majeurs, on répondra 3\coeur\ sur le check back stayman (3\carreau\ est naturel).\\

		\SSection{Convention Bessis}
		\begin{tabular}{cccc|l}
		1\trefle/\carreau & - & 2\coeur && 6-10 ; 5\pique/4\coeur\ (Possiblement 5/5)\\
		\end{tabular}\\
		ATTENTION : Par inférence, 1\trefle/\carreau\ - 1\pique\ - 1SA - 2\coeur\ promet un 5/4 plus faible\\
	
		\begin{tabular}{cccc|l}
		1\trefle/\carreau & - & 2\coeur & - &\\
		2SA & - & 3\trefle && 5/4 minimum\\
		&& 3\carreau && 5/5 minimum\\
		&& 3\coeur && 5/4 maximum\\
		&& 3\pique && 5/5 maximum\\
		\end{tabular}\\\\
	
		La séquence 1\trefle/\carreau\ - 2\pique\ promet par contre 6\pique\ dans une main très faible\\

		\SSection{Double 2}
		\begin{tabular}{cccc|l}
		1\trefle/\carreau & - & 1\pique & - &\\
		1SA & - & 2\trefle & - & Relais\\
		2\carreau & - & Passe && Pour jouer\\
		&& 2\coeur && Main limite avec 5\pique/4\coeur\\
		&& 2\pique && Main limite avec 5\pique\\
		&& 2SA && Main limite avec du \trefle\\
		&& 3\trefle && Faible avec du \trefle\\
		&& 3\carreau && Main limite avec du \carreau\\
		&& 3\coeur && 6\pique/4\coeur\ Main limite\\
		&& 3\pique && 6\pique\ Main limite\\
		&& 3SA && Proposition entre 3SA et 4\pique\\
		\end{tabular}\\
		Même principe sur 1\trefle/\carreau\ - 1\coeur\\
		
		\begin{tabular}{cccc|l}
		1\trefle/\carreau & - & 1\pique & - &\\
		1SA & - & 2\carreau & - & Relais FM (Soit 5\pique ; Soit un problème d'arrêt)\\
		2\coeur &&&& 4\coeur\ (Ne dénie pas 3\pique)\\
		2\pique &&&& 3\pique\\
		2SA &&&& Arrêt \coeur\ et \carreau/\trefle\\
		3\trefle/\carreau &&&& Belle couleure, mauvais arrêts\\
		3\carreau/\trefle &&&& Arrêt pour SA\\
		\end{tabular}\\
		Même principe sur 1\trefle/\carreau\ - 1\coeur\ et 1\coeur\ - 1\pique\\

		Le double deux est toujours valable si l'adversaire était intervenu sur l'ouverture. \\

		\begin{tabular}{cccc|l}
		1\coeur & - & 1\pique & - &\\
		1SA & - & 2\trefle & - & Relais\\
		2\carreau & - & Passe && Pour jouer\\
		&& 2\coeur && Main limite avec 5\pique\ et Honneur second \coeur\\
		&& 2\pique && Main limite avec 5\pique\ (ou 6 moches)\\
		&& 2SA && Main limite avec du \trefle\\
		&& 3\trefle && Faible avec du \trefle\\
		&& 3\carreau && Main limite avec du \carreau\\
		&& 3\pique && Main limite avec 6\pique\\
		&& 3SA && Proposition entre 3SA et 4\pique\\
		\end{tabular}\\\\
	
		\SSection{Troisième forcing}
		\begin{tabular}{cccc|l}
		1\trefle & - & 1\coeur & - &\\
		2\trefle & - & 2\carreau & - & Troisième forcing\\
		2\coeur &&&& 3\coeur\ minimum (Parfois 2\coeur\ sans autre enchère)\\
		2\pique/3\carreau &&&& 14+ ; Un arrêt pour 3SA\\
		2SA &&&& Minimum ; Tous les arrêts\\
		3\trefle &&&& Minimum\\
		3\coeur &&&& 14+ ; 3\coeur\\
		3SA &&&& 14+ ; Tous les arrêts\\
		\end{tabular}\\
		De même pour la séquence 1\carreau\ - 1\pique\ - 2\carreau\ - 2\coeur\\
			
		\begin{tabular}{cccc|l}
		1\trefle & - & 1\pique & - &\\
		2\trefle & - & 2\carreau && Troisième forcing sans 4\coeur\\
		&& 2\coeur && Troisième forcing avec 4\coeur\\
		\end{tabular}\\\\
			
		\begin{tabular}{cccc|l}
		1\trefle & - & 1\coeur & - &\\
		2\trefle & - & 2\carreau & - & Troisième forcing\\
		2\coeur & - & 2SA && Relais propositionnel (et vérifiant le fit)\\
		&& 3\coeur &&Chelemisant\\
		\end{tabular}\\
		De même sur les autres séquences\\

		\SSection{2SA Modérateur}
		Joué sur les séquences 1\trefle/\carreau\ 1\coeur/\pique,  mais pas 1\trefle/\carreau\ 1SA\\

		\SSection{1\trefle/\carreau\ 1Majeur\ 3SA}
		\begin{tabular}{cccc|l}
		1\trefle & - & 1\coeur/1\pique &&\\
		3SA &&&& 18-19 régulier avec le fit \coeur/\pique\ (5422, 4432, 4333)\\
		4\coeur/4\pique &&&& 6\trefle\ et 4\coeur/\pique\ trop faible pour dire 4\trefle\\
		\end{tabular}\\\\

		\SSection{1\trefle/\carreau\ 1\coeur\ 1\pique\ 1SA}
		\begin{tabular}{cccc|l}
		1\trefle/\carreau & - & 1\coeur & - &\\
		1\pique  & - & 1SA &&\\
		2\carreau/\trefle &&&& 3\coeur\ 12-14 irrégulier ou 18-19\\
		2\coeur &&&& 3\coeur\ 15-17\\
		\end{tabular}\\\\

		\SSection{1\carreau\ 1Majeur\ 2\trefle\ 3\carreau}
		\begin{tabular}{cccc|l}
		1\carreau & - & 1\coeur/1\pique & - &\\
		2\trefle  & - & 3\carreau && Soutien limite 9-11 NF (Le fit différé passe par la quatrième forcing)\\
		\end{tabular}\\\\

\newpage
	\Section{En cas d'interventions}
		\SSection{Cachalot}
		\begin{tabular}{cccc|l}
		1\trefle & X & XX && Du jeu\\
		&& 1\carreau && Texas 4-5\coeur\\
		&& 1\coeur && Texas 4-5\pique\\
		&& 1\pique && Texas SA\\
		&& 1SA/2\trefle/2\carreau/2\coeur && Texas 6 cartes (Fort ou faible)\\
		&& 2\pique && Truscott irrégulier\\
		&& 2SA && Truscott régulier\\
		&& 3\trefle && Barrage\\
		&& 3\carreau/3\coeur && Texas 6 cartes limite\\
		\end{tabular}\\
		De même sur l'ouverture d'1\carreau\ (Attention: 3\trefle\ devient texas \coeur\ car 3\carreau\ est naturel)\\

		\begin{tabular}{cccc|l}
		1\trefle & 1\carreau & X && Texas 4-5\coeur\\
		&& 1\coeur && Texas 4-5\pique\\
		&& 1\pique && Texas SA\\
		&& 1SA/2\carreau/2\coeur && Texas 6 cartes (Fort ou faible)\\
		&& 2\trefle && Texas impossible = Cue-bid\\
		&& 2\pique && Truscott sans l'arrêt \carreau\\
		&& 2SA && Truscott avec l'arrêt \carreau\\
		&& 3\trefle && Barrage\\
		&& 3\carreau/3\coeur && Texas 6 cartes limite\\
		\end{tabular}\\
		De même pour l'intervention d'1\coeur\ sur 1\trefle/\carreau\\

		Cela fonctionne également sur les interventions en texas
		\begin{itemize}
		\item 1\trefle\ 1\carreau\ (Texas coeur): X = Du carreau / 1\coeur\ = 4-5\pique / ...
		\item 1\trefle\ 1\coeur\ (Texas pique): X = 4-5\coeur / 1\pique\ = Texas 1SA / ...\\
		\end{itemize}

		\begin{tabular}{cccc|l}
		1\trefle & 1\carreau & X & - & Texas 4-5\coeur\\
		1\coeur &&&& 3\coeur\ sans l'arrêt (Ne dénie pas 4\pique)\\
		1\pique &&&& 4\pique\ sans l'arrêt ni 3\coeur\\
		1SA &&&& Peut cacher 4\pique\ ou 3\coeur\ avec l'arrêt \carreau\\
		\end{tabular}\\\\

		\begin{tabular}{cccc|l}
		1\trefle & 1\coeur & X & - & Texas 4-5\pique\\
		1\pique &&&& 3\pique\ sans l'arrêt\\
		1SA &&&& Peut cacher 3\pique\ avec l'arrêt \coeur\\
		\end{tabular}\\\\
		
		\SSection{Rodrigue}
		\begin{tabular}{cccc|l}
		1\trefle/\carreau & 1\pique & X && Spoutnik ; Dénie 5\coeur\\
		&& 2\trefle/\carreau && 8-10 ; 5\coeur\ (Rodrigue)\\
		&& 3\trefle/\carreau && Barrage\\
		\end{tabular}\\\\

		\begin{tabular}{cccc|l}
		1\carreau & 2\trefle & X && 8-10 ; 5\pique\ OU 4\coeur/4\pique\ OU 4\coeur/\pique\ et du \carreau\ OU 11+\\
		&& 2\carreau && 8-10 ; 5\coeur\ (Rodrigue)\\
		&& 3\carreau && Barrage\\
		\end{tabular}\\\\

\Chap{Interventions \& Défenses}
	\Section{Sur 1SA Fort}
	\underline{En 2$^{ème}$ position}
	
	\begin{tabular}{cccc|l}
	1SA & X &&& Mineur cinquième et majeure quatrième\\
	& 2\trefle &&& Landy 5/4 majeur (2\carreau\ demande de la plus longue)\\
	& 2\carreau &&& Multi (Développement comme sur un Multi d'ouverture)\\
	& 2\coeur/2\pique &&& 5\coeur/\pique\ et une mineure 4$^{ème}$ (Développement comme sur un Muiderberg)\\
	& 2SA &&& Bicolore 5\trefle/5\carreau\\
	& 3\trefle/3\carreau &&& Naturel (Tendance barrage)\\
	& 3\coeur/3\pique &&& Barrage\\
	\end{tabular}\\
	Sur le X, 2\trefle\ demande la mineure et 2\carreau\ demande la majeure\\
	
	\underline{En 4$^{ème}$ position}\\
	Tout naturel et X pour les majeurs (souplement transformable)\\

	\underline{Après un Texas}
	
	\begin{itemize}
	\item X = D'entame (Le contre est d'appel en réveil après la rectification)
	\item Rectification = Bicolore 5/4 majeur/mineur
	\item 2SA = Bicolore 5/5 mineur (De même en réveil après la rectification)
	\item Autre = Unicolore naturel (De même en réveil après la rectification)\\
	\end{itemize}

	\underline{Cas de l'intervention adverse par 1SA :} Landik amélioré\\
	\begin{tabular}{cccc|l}
	1\trefle/\carreau & 1SA & X && Punitif\\
	&& 2\trefle && Landik (5/4 majeur)\\
	&& 2\carreau/2\coeur/2\pique/3\trefle && Texas\\
	&& 2SA && Cue bid fitté\\
	\end{tabular}\\\\

	\begin{tabular}{cccc|l}
	1\coeur & 1SA & X && Punitif\\
	&& 2\trefle && Texas \carreau\\
	&& 2\carreau && Texas \pique\ (car 2\coeur\ est naturel)\\
	&& 2\coeur && Fit\\
	&& 2\pique && Texas \trefle\\
	&& 2SA && Cue bid fitté\\
	\end{tabular}\\
	Même principe sur l'ouverture d'1\pique\\

	\Section{Sur 1SA Faible}
	\begin{tabular}{cccc|l}
	1SA & X &&& Jeu régulier ou Unicolore puissant\\
	& 2\trefle &&& Landy\\
	& 2\carreau/2\coeur/2\pique/3\trefle &&& Texas\\
	& 2SA &&& Bicolore mineur\\
	\end{tabular}\\
	Le X promet 1 point de plus que le maximum de la zone du SA faible (15 H vs 12-14 ; 13 H vs 10-12)\\
	Sur le Texas, la rectification indique un jeu faible et ne promet pas le fit\\

	\begin{tabular}{cccc|l}
	1SA & X & - & Passe & Régulier sans majeure 5$^{ème}$\\
	&&& 2\trefle/2\carreau/2\coeur/2\pique & 0-8 ; Naturel\\
	&&& 2SA/3\trefle/3\carreau/3\coeur & 9+ ; Texas\\
	\end{tabular}\\\\

	\begin{tabular}{cccc|l}
	1SA & 2\trefle & - & 2\carreau & Relais\\
	&&& 2\coeur/2\pique & Préférence\\
	&&& 2SA & Cue-bid\\
	&&& 3\coeur/3\pique & Proposition\\
	&&& 4\coeur/4\pique & Pour les jouer\\
	\end{tabular}

	\Section{Sur 2\coeur/2\pique\ faible}
	\begin{tabular}{cccc|l}
	2\coeur & X & - & 2\pique & 0-7 naturel\\
	&&& 2SA & Mini cue-bid (8-10 sans 5\pique OU Main de manche avec 4\pique)\\
	&&& 3\trefle/3\carreau & 0-7 naturel\\
	&&& 3\coeur & Main de manche sans l'arrêt \coeur\ ni 4\pique\\
	&&& 3\pique & 8-10 avec 5\pique\\
	&&& 3SA & Manche avec l'arrêt \coeur sans 4\pique\\
	\end{tabular}\\
	Sur 2SA,  le contreur nommera sa meilleure mineure dans la zone 12-15\\

	\begin{tabular}{cccc|l}
	2\coeur & X & - & 2SA &\\
	- & 3\trefle & - & Passe/3\carreau/3\pique & Naturel 8-10\\
	 &&& 3\coeur & Main de manche sans l'arrêt \coeur\ avec 4\pique\\
	 &&& 3SA & Main de manche avec l'arrêt \coeur\ et 4\pique\\
	\end{tabular}\\\\

	\Section{Sur 2\carreau\ Multi}
	\begin{tabular}{cccc|l}
	2\carreau & X &&& Contre d'appel court à \pique\\
	& 2\coeur &&& Contre d'appel court à \coeur\\
	& 2\pique/2SA/3\trefle/3\carreau &&& Naturel\\
	& 3\coeur/3\pique &&& Naturel 6+\coeur/\pique\\
	& 3SA &&& Tendance Gambling\\
	& 4\trefle/4\carreau &&& Bicolore 5\pique\ et 5\trefle/\carreau\\
	\end{tabular}\\
	Sur X/2\coeur\ la réponse de 2SA est mini-cuebid\\
	
	\Section{Sur 2\coeur\ Bicolore majeur}
	\begin{tabular}{cccc|l}
	2\coeur & X &&& Contre d'appel avec l'arrêt \coeur\\
	& 2\pique &&& Contre d'appel avec l'arrêt \pique\\
	& Autre &&& Naturel\\
	\end{tabular}\\\\

	\Section{Sur 1\trefle\ Fort}
	\begin{tabular}{cccc|l}
	1\trefle & -  & X && Bicolore de même Couleure\\
	&& 1\carreau && Bicolore de même Rang\\
	&& 1SA && Bicolore Mélangé\\
	\end{tabular}\\\\

	\Section{Michaëls}
	On joue les michaëls précisés sauf :\\
		
	\begin{tabular}{cccc|l}
	1\trefle & 2\trefle &&& Bicolore majeur\\
	& 2\carreau &&& Barrage naturel\\
	\end{tabular}\\\\

	1\trefle\ 3\trefle\ et 1\carreau\ 3\trefle\ sont Michaël.\\

	\Section{Défense contre les Michaëls}
	Le principe général est le suivant :
	\begin{itemize}
	\item Cue bid de la moins chère : La couleure non nommée, FM
	\item Cue bid de la plus chère : Fittée, FM
	\item Autre : Naturel NF\\
	\item X : Du jeu (Si X encore, punitif)
	\item Passe puis X : Réveil\\
	\end{itemize}
	Fonctionne également sur 1\trefle/\carreau\ 2\carreau

	\Section{Principes d'interventions}
	\begin{itemize}
	\item 1/1 forcing
	\item 2/1 forcing (sauf après passe d'entrée ou enchère du numéro 3 qui n'est pas un fit faible)
	\item 2/2 forcing (sauf après passe d'entrée ou enchère du numéro 3 qui n'est pas un fit faible)\\
	\end{itemize}
	
	Exemple:
	\begin{itemize}
	\item 1\trefle\ 1\pique\ 1SA 2\carreau\ = Non forcing
	\item 1\trefle\ 1\pique\ X 2\carreau\ = Non forcing
	\item 1\trefle\ 1\pique\ 2\coeur\ 3\carreau\ = Non forcing (La 4e couleur est NF sauf au palier de 1)
	\item 1\trefle\ 1\pique\ 2\trefle\ 2\carreau\ = Forcing (Le changement de couleur n'est forcing qu'en cas de fit faible)\\
	\end{itemize}

	\underline{Développements}\\
	\begin{tabular}{cccc|l}
	1\carreau & 1\coeur\  & - & 2\carreau\ & Fitté au moins limite par 3 OU Une main forte sans enchère\\
	&&& 2\coeur & Fit classique\\
	&&& 2SA & Mixed raise\\
	&&& 3\carreau\ & Fitté au moins limite par 4\\
	&&& 3\coeur\ & Barrage\\
	\end{tabular}\\
	Cela fonctionne également en cas d'enchère (ou de X) du numéro 3\\
	1\carreau\ 1\coeur\ X XX sera un honneur à \coeur\\

	\begin{tabular}{cccc|l}
	1\trefle & 1\carreau/1\coeur  & - & 1\pique\ & (Dès 4\pique)\\
	- & 2\trefle &&& 3\pique\ avec l'ouverture OU Une main forte sans enchère\\
	& 2\pique &&& 3\pique\ sans l'ouverture\\
	& 3\trefle &&& 4\pique\ avec l'ouverture\\
	& 3\pique &&& 4\pique\ sans l'ouverture\\
	\end{tabular}\\\\

	\begin{tabular}{cccc|l}
	1\trefle & 1\carreau & X (Du \coeur) & XX & Un honneur à \carreau\\
	&&& 1\coeur & Rectification = D'appel (Souvent 4\pique)\\
	&&& 1\pique & 5\pique\\
	\end{tabular}\\\\

	\begin{tabular}{cccc|l}
	1\trefle & 1\carreau & 1\coeur\ (Du \pique) & X & 5\coeur\\
	&&& 1\pique & Rectification = D'appel (Souvent 4\coeur)\\
	\end{tabular}\\\\

	\begin{tabular}{cccc|l}
	1\trefle & 1\coeur\  & - & - &\\
	X & - & - & XX & SOS (puis Baron)\\
	&&& 1\pique/2\trefle/2\carreau & Naturel\\
	&&& 1SA & Bicolore mineur\\
	\end{tabular}\\\\

\newpage
	\Section{Séquences à 4}
	\begin{itemize}
	\item 1\trefle\ Passe 1\pique\ 1SA = 16-18 régulier
	\item Passe 1\trefle\ Passe 1\pique\ 1SA = 6\carreau\ 4\coeur\\

	\item 1\trefle\ 1\pique\ Passe 2\trefle\ X = Demande l'entame (car la couleur n'a pas été promis 5$^{ème}$)
	\item 1\coeur\ 1\pique\ Passe 2\coeur\ X = Demande une autre entame (car la couleur a été promise 5$^{ème}$)\\

	\item 1\carreau\ Passe 1\coeur\ X XX = 3\coeur\ dans une main positive
	\item 1\carreau\ Passe 1\coeur\ 1\pique\ X = 3\coeur\ dans une main positive

	\item 1\coeur\ Passe 2\coeur\ 2\pique\ X = Punitif
	\item 1\coeur\ Passe 2\coeur\ 3\trefle\ X = Punitif (3\carreau\ sera d'essai généralisé)
	\item 1\coeur\ Passe 2\coeur\ 3\carreau\ X = Enchère d'essai (Car pas manque de place)\\

	\item 1\pique\ X 2\pique\ 3\coeur\ = 5 cartes
	\item 1\pique\ X 2\pique\ 2SA Passe 3\trefle/\carreau\ Passe 3\coeur\ = 4 cartes
	\item 1\pique\ Passe 2\pique\ Passe Passe X Passe 3\trefle/\carreau/\coeur\ = 5 cartes
	\item 1\pique\ Passe 2\pique\ Passe Passe X Passe 2SA = Baron\\

	\item 1\coeur\ 1SA Passe Passe X = Du jeu et 6\coeur/4\pique
	\item 1\trefle\ 1SA Passe Passe X = Appel 5431 court à \carreau\ (relais à 2\carreau\ nomme ta majeur)\\

	\item 1\pique\ Passe 1SA 2\carreau\ X = Appel
	\item 1\carreau\ Passe 1SA 2\pique\ X = Appel\\

	\item 1\pique\ Passe Passe X 2\pique\ X = Appel
	\item 1\pique\ Passe Passe X 2\carreau\ X = Punitif (Car deux couleurs nommées)
	\item 1\pique\ Passe Passe 2\trefle\ 2\pique\ X = Appel
	\item 1\pique\ Passe Passe 2\trefle\ 2\carreau\ X = Punitif (Car deux couleurs nommées)
	\item 1\carreau\ Passe 1SA X 2\carreau\ X = Appel
	\item 1\carreau\ Passe 1\coeur\ X 2\carreau\ X = Punitif (Car deux couleurs nommées)
	\item 1\carreau\ Passe 1SA 2\pique\ 3\carreau\ X = Appel
	\item 1\carreau\ Passe 1SA 2\pique\ 3\trefle\ X = Punitif (Car deux couleurs nommées)\\

	\item 1\coeur\ Passe 1\pique\ Passe 2\coeur\ X = Appel
	\item 1\coeur\ Passe 1SA Passe 2\coeur\ X = Punitif
	\item 1\coeur\ Passe 1SA Passe 2\trefle\ X  = Appel
	\item 1\coeur\ Passe 1SA Passe 2\trefle\ Passe 2\coeur\ Passe Passe X  = Punitif\\

	\item 1\carreau\ Passe 1\coeur\ 2\coeur\ X = Appel
	\item 1\carreau\ Passe 1\coeur\ 2\coeur\ Passe Passe X = Appel\\

	\item 1\coeur\ 3\pique\ 4SA = BW \coeur
	\item 1\coeur\ 4\carreau\ 4SA = BW \coeur
	\item 1\pique\ 4\coeur\ 4SA = BW \pique
	\item 1\coeur\ 4\pique\ 4SA = Bicolore mineure ou beau fit coeur\\
	
	\item 1\trefle\ 1\coeur\ Passe 2\coeur\ X Passe 2SA = Pour les mineures
	\item 1\trefle\ 2\coeur\ Passe Passe X Passe 2SA = Pour les mineures\\

	\item 1\pique\ X Passe 2SA = 8-10 avec 4\coeur
	\end{itemize}

\Chap{Nota bene}
	\Section{Blackwood et contrôle}
	\begin{itemize}
	\item Blackwood 5 clés 41/30
	\item Blackwood d'exclusions 0, 1, 2, 3
	\item En cas d'intervention,  C0p1 (Contre = 30,  Passe = 41)\\
	\end{itemize}

	Dans une séquence chelemisante,  lorsqu'un contrôle est sauté,  le prochain contrôle promet la couleur sautée (mais ne promet pas forcément le contrôle nommé):\\
	\begin{tabular}{cccc|l}
	1\pique & -  & 2\trefle & - &\\
	2\pique & - & 3\pique & - &\\
	4\carreau & - & 4\coeur && 4\coeur\ = Contrôle \trefle\ avec ou sans le contrôle \coeur\\
	\end{tabular}\\\\

	\Section{Enchère d'essai}
	\begin{tabular}{cccc|l}
	1\pique & -  & 2\pique & - &\\
	2SA & - & 3\trefle/3\carreau/3\coeur && Maximum,  un gros honneur\\
	&& 3\pique && Minimum\\
	&& 4\trefle/4\carreau/4\coeur && Maximum,  une courte\\
	\end{tabular}\\
	Les réponses sont les mêmes sur 1\trefle/\carreau\ 1\pique\ 2\pique\ 2SA\\

	\begin{tabular}{cccc|l}
	1\pique & -  & 2\pique & - &\\
	3\trefle & - & 3\carreau/3\coeur && Une force\\
	&& 3\pique && Minimum\\
	&& 4\trefle && Un fit par 4 cartes\\
	&& 4\carreau/4\coeur && Maximum,  une courte\\
	\end{tabular}\\\\

	\Section{Flanc}
		\SSection{SA}
		\begin{itemize}
		\item 4$^{ème}$ meilleure ou Top of nothing
		\item Tête de séquence
		\item Gros appel sur ADV
		\item Roi = Demande de déblocage (ou gros appel si singleton au mort)
		\item Défausse pair/impair,  gros appel si urgence
		\item Switch: Petit prometteur (ou Pair/impair selon)\\
		\end{itemize}

		\SSection{Couleur}
		\begin{itemize}
		\item Tête de séquence
		\item R dans AR sec ou avec un singleton annexe
		\item Sur l'As,  gros appel avec un doubleton ou la Dame (mais petit dans D seconde)
		\item Sur l'As,  appel/refus avec deux doubletons au mort (pour switch ou non dans l'autre doubleton)
		\item Sur l'As,  préférentielle si singleton au mort (ou singleton détecté chez le déclarant)
		\item Sur le Roi,  gros appel avec le V ou l'As pour éviter le coup de Bath
		\item Pair impair inversé à l'atout
		\item Défausse pair/impair,  gros appel si urgence
		\item Switch: Petit prometteur (ou Pair/impair selon)\\
		\end{itemize}
		NB: Dans RV109, R109 ou RV10,  on switch de la plus petite (i.e.  le 9 ou le 10)
\end{document}