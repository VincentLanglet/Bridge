\documentclass[a4paper, oneside, 11pt]{report}
\usepackage{Bridge}

\begin{document}
\Chap{Ouverture de 1SA}
	\underline{Cas de l'intervention par 1SA :} On ignore l'ouverture
	
	En cas de contre punitif, on joue naturel :\\
	\begin{tabular}{cccc|l}
	1X & 1SA & X & Passe & 4333\\
	&&& XX & 4432 ou 4441 (Puis Baron)\\
	&&& 2\trefle/2\carreau/2\coeur/2\pique & Naturel\\
	\end{tabular}\\
	
	\underline{Cas de l'intervention adverse par 1SA :} Landik\\

	\Section{Développements}
		\SSection{Réponses spéciales}
		\begin{tabular}{cccc|l}
		1SA & - & 3\carreau && Stayman FM 4333\\
		&& 4\trefle && Bicolore majeur au moins 6/5\\
		&& 4\carreau && Bicolore majeur 5/5\\
		\end{tabular}\\\\

		Après un Texas Mineur, on nomme les singletons majeurs naturellement.\\

		\SSection{Autres}	
		\begin{tabular}{cccc|l}
		1SA & - & 4\trefle & - & \it{Bicolore majeur au moins 6/5}\\
		4\carreau &&&& Relais pour connaître la majeure la plus longue\\
		4\coeur/4\pique &&&& Pour jouer\\
		\end{tabular}\\\\

	\Section{Réaction face aux interventions}
		\SSection{Contre du Stayman/Texas}
		Si le Texas est contré, on fait jouer de l'autre main :
		\begin{itemize}
		\item Passe = Non fitté
		\item XX = Fitté\\
		\end{itemize}
		
		Si le 2\trefle\ Stayman est contré :\\
		\begin{tabular}{cccc|l}
		1SA & - & 2\trefle & X &\\
		Passe &&&& Les autres mains (Typiquement des 4333)\\
		XX &&&& Pour jouer\\
		2\carreau &&&& Naturel\\
		2\coeur/\pique &&&& 4\pique/\coeur\ (Inversé pour mettre le X à l'entame)\\
		2SA &&&& Un arrêt solide\\
		3\trefle &&&& Les deux majeurs (Pas de Texas \& Le palier de 3 est chelemisant)\\
		\end{tabular}\\\\
		
		\begin{tabular}{cccc|l}
		1SA & - & 2\trefle & X &\\
		3\trefle & - & 3\carreau && Propositionnel (Sur 3\coeur\ minimum, 3\pique\ est pour jouer)\\
		&& 3\coeur/3\pique && Chelemisant\\
		&& 4\coeur/4\pique && Pour jouer\\
		\end{tabular}\\\\
		
		\SSection{Autres}
		\begin{tabular}{cccc|l}
		1SA & 2\trefle\ (Landy) & X && Punitif dans au moins une majeure OU 8H+ sans arrêt\\
		&& 2\coeur/2\pique\ && Singleton \coeur/\pique\ avec un 54 mineur\\
		&& 2SA/3\trefle && Texas \trefle/\carreau\\
		&& 3\carreau && Une majeure cinquième\\
		&& 3\coeur/3\pique && Chicane \pique/\coeur\ avec un 55 mineur\\
		\end{tabular}

\Chap{Ouverture de 2\trefle/2\carreau/2\coeur/2\pique/2SA et 3SA}
	\Section{2\trefle\ Bivalent}
	2\trefle\ regroupe les mains régulières 23+, les deux forts en majeur et les mains FM.\\
	
	\begin{tabular}{cccc|l}
	2\trefle & - & 2\carreau && Relais FM\\
	&& 2\coeur && Pour jouer en face d'un 2\coeur\ fort\\
	&& 2\pique && Pour jouer en face d'un 2\pique\ fort\\
	&& 3\trefle && Jeu faible, au moins 5-4 en majeur (Stayman)\\
	&& 3\carreau && Jeu faible, 4-4 en majeur (Stayman)\\
	\end{tabular}\\\\
	
	\begin{tabular}{cccc|l}
	2\trefle & - & 3\trefle &&\\
	3\carreau &&&& Demande d'une majeur cinquième (réponse en chassé croisé)\\
	3\coeur/3\pique/4\trefle/4\carreau &&&& Naturel chelemisant\\
	3SA &&&& Pour jouer\\
	4\coeur/4\pique/5\trefle/5\carreau &&&& Pour jouer\\	
	\end{tabular}\\
	Même principe pour la séquence 2\trefle\ - 3\carreau\\

	\Section{2\carreau\ Multi}
	Version faible : 2\carreau\ est un unicolore majeur faible indéterminé\\

	\begin{tabular}{cccc|l}
	2\carreau & - & 2\coeur/2\pique/3\coeur/3\pique && Passe ou Corrige\\
	&& 2SA && Relais\\
	&& 3\trefle && Naturel ; Pour jouer\\
	&& 3\carreau && Forcing avec l'autre majeure présumée\\
	&& 4\trefle && Nomme ta majeur en Texas\\
	&& 4\carreau && Nomme ta majeur\\
	&& 4\coeur/4\pique && Naturel\\
	\end{tabular}\\\\

	\begin{tabular}{cccc|l}
	2\carreau & - & 2SA & - &\\
	3\trefle/3\carreau &&&& Minimum \coeur/\pique\ (3\carreau/\coeur\ pour vérifier qu'il s'agit d'un vrai barrage)\\
	3\coeur/3\pique &&&& Maximum \pique/\coeur\\
	\end{tabular}\\\\
	
	\begin{tabular}{cccc|l}
	2\carreau & - & 3\carreau & - &\\
	3\coeur &&&& Unicolore \coeur\ sans 3\pique\\
	3\pique &&&& Unicolore \pique\ sans 3\coeur\\
	4\trefle &&&& Unicolore \coeur\ avec 3\pique\\
	4\carreau &&&& Unicolore \pique\ avec 3\coeur\\
	\end{tabular}\\\\

	\begin{tabular}{cccc|l}
	2\carreau & X & Passe && Du carreau ; Accepte 2\carreau\ X\\
	&& XX && Pour jouer 2\coeur\ ou 2\pique\\
	&& 2\coeur && Pour jouer 2\coeur\ ou 3\pique\\
	&& 2\pique/3\coeur/3\pique && Passe ou Corrige\\
	&& 4\trefle && Nomme ta majeure en Texas\\
	&& 4\carreau && Nomme ta majeure\\
	&& 4\coeur/4\pique && Naturel\\
	\end{tabular}\\\\
	En cas d'intervention naturelle majeure, le contre est passe ou corrige\\	
	En cas d'intervention naturelle par 3\trefle, les enchères 4\trefle\ et 4\carreau\ gardent leur sens\\	
	En cas d'intervention naturelle par 3\carreau, 4\trefle\ est non forcing et 4\carreau\ demande la majeur\\

\newpage
	\Section{2\coeur\ \& 2\pique\ Muiderberg}
	2\coeur/\pique\ est un bicolore faible au moins 5\coeur/\pique\ et 4\trefle/4\carreau\\

	\begin{tabular}{cccc|l}
	2\pique && 2SA && Relais descriptif\\
	&& 3\trefle/4\trefle/5\trefle && Passe ou Corrige\\
	&& 3\coeur && 6\coeur\ forcing\\
	&& 3\pique/4\pique && Barrage\\
	&& 3SA && Pour jouer\\
	\end{tabular}\\\\
	
	Même principe sur 2\coeur\ avec les enchères:
	\begin{tabular}{cccc|l}
	2\coeur && 2\pique && Passe (doubleton) ou Corrige\\
	&& 3\pique && 6\pique\ forcing\\
	\end{tabular}\\\\

	\begin{tabular}{cccc|l}
	2\coeur/\pique & - & 2SA & - &\\
	3\trefle/3\carreau &&&& 5\coeur/\pique\ et 4\trefle/4\carreau\ Minimum\\
	3\coeur/3\pique &&&& 5\coeur/\pique\ et 4\trefle/4\carreau\ Maximum\\
	4\trefle/4\carreau &&&& 5\coeur/\pique\ et 5\trefle/5\carreau\ Maximum\\
	\end{tabular}\\\\
	
	Sur les réponses de 3\trefle/3\carreau/3\coeur/3\pique:
	\begin{itemize}
	\item 3\coeur/\pique\ est pour les jouer
	\item 4\trefle/\carreau\ est chelemisant dans la mineure
	\item Le reste est chelemisant dans la majeure\\
	\end{itemize}
	
	Après X ou sur une ouverture en $3^{ème}$, 2SA devient un relais pour la mineur et 3\trefle/\carreau\ sont naturels\\

	\Section{2SA}
	2SA annonce 20-22HL. Les réponses sont standards. La rectification du Texas est fittée.\\

	\Section{3SA Gambling}
	\begin{tabular}{cccc|l}
	3SA & - & 4\trefle && Passe/Corrige\\
	&& 4\carreau && Demande de singleton\\
	&& 4\coeur/4\pique/5\trefle/5\carreau/6\trefle/6\carreau && Pour jouer\\
	&& 4SA && Nomme ta mineure\\
	\end{tabular}\\\\
	
	\begin{tabular}{cccc|l}
	3SA & - & 4\carreau & - &\\
	4\coeur &&&& Singleton \coeur\\
	4\pique &&&& Singleton \pique\\
	4SA &&&& 7222\\
	5\trefle &&&& Singleton \carreau\ donc avec les \trefle\\
	5\carreau &&&& Singleton \trefle\ donc avec les \carreau\\
	\end{tabular}\\\\
	
\Chap{Ouverture majeure}
	\Section{Fit majeur (Sans passe d'entrée)}
	\begin{tabular}{cccc|l}
	1\coeur & - & 1SA && Peut cacher un fit faible\\
	&& 2\coeur && 6-10 ; Fitté par 3\\
	&& 2\pique && Barrage 6\pique\\
	&& 2SA && Fitté FM ; Main intéressante\\
	&& 3\trefle && Fitté limite par 4\\
	&& 3\carreau && Fitté limite par 3\\
	&& 3\coeur && 6-10 ; Fitté par 4\\
	&& 3\pique/4\trefle/4\carreau && Splinter\\
	&& 3SA && Fitté FM ; Main banale\\
	&& 4\coeur && Barrage\\
	\end{tabular}\\\\

	Même principe sur 1\pique, sauf l'enchère de 3\coeur, qui est une main limite, sans le fit \pique, avec 6\coeur\\

		\SSection{En cas d'intervention}
		Sur le contre, on jouera 2SA Truscott, 3SA Super-Truscott\\
		De plus, on distingue deux types de fit au palier de 2 :

		\begin{tabular}{cccc|l}
		1\coeur & X & 2\carreau && Fit constructif\\
		&& 2\coeur && Fit barrage\\
		\end{tabular}
		
		De même pour les \pique\\

		En cas d'intervention à la couleur, on jouera 2SA Truscott, le Cue-bid fitté par 4
		
		Si X ou intervention, les changements de couleur à saut sont naturels faibles non fitté. Sauf 4\trefle\ et 4\carreau\\

		\SSection{Développement après 1\coeur/\pique\ - 3\trefle}
		\begin{tabular}{cccc|l}
		1\coeur/\pique & - & 3\trefle & - &\\
		3\carreau &&&& Relais demande de courte\\
		\end{tabular}\\\\
		
		\begin{tabular}{cccc|l}
		1\coeur/\pique & - & 3\trefle & - &\\
		3\coeur/3\pique/3SA &&&& Les courtes dans l'ordre\\
		4\trefle/4\carreau &&&& Controle sans courte\\
		\end{tabular}\\\\

		\SSection{2SA Forcing de manche}
		Le principe sera le même sur l'ouverture d'1\pique\\
		
		\begin{tabular}{cccc|l}
		1\coeur & - & 2SA & - &\\
		3\trefle &&&& 12-14\\
		3\carreau &&&& 15+ sans courte\\
		3\coeur/3\pique/3SA &&&& 15+ les courtes dans l'ordre\\
		4\trefle/4\carreau &&&& Bicolore concentré\\
		\end{tabular}\\\\
		
		\begin{tabular}{cccc|l}
		1\coeur & - & 2SA & - &\\
		3\trefle & - & 3\carreau & - &\\
		3\coeur/3\pique/3SA &&&& Les courtes dans l'ordre\\
		4\trefle/4\carreau &&&& Controle sans courte\\
		\end{tabular}\\
	
\newpage
	\Section{Fit majeur (Après passe d'entrée)}
		\SSection{2\trefle\ Drury et inférence}
		Cette enchère regroupe les fits limites\\
		\begin{tabular}{cccc|l}
		& (-) & - & - &\\
		1\coeur/\pique & - & 2\trefle & - &\\
		2\carreau &&&& Demande de description\\
		2\coeur/\pique &&&& Pour jouer\\
		4\coeur/\pique &&&& Pour jouer\\
		\end{tabular}\\
		Après une ouverture d'1\pique, 2\coeur\ promet un jeu sans l'ouverture dans un 5/4\\

		\begin{tabular}{cccc|l}
		& (-) & - & - &\\
		1\coeur/\pique & - & 2\trefle & - &\\
		2\carreau & - & 2\coeur/\pique && Main limite fittée par 3 cartes\\
		&& 2SA && 4333 fitté par 4 cartes\\
		&& 3\coeur/\pique && Main limite fittée par 4 cartes\\
		\end{tabular}\\\\
		
		L'enchère de 2SA annonce alors une main limite fitté par 4 cartes avec un singleton
		
		La demande de singleton se fait à 3\trefle\ on répond les courtes dans l'ordre\\

	\Section{Séquences particulières}
		\SSection{Cas des coeurs}
	
		Par inférence de 1\pique\ - 3\coeur, la séquence 		
		\begin{tabular}{cccc|l}
		1\pique & - & 2\coeur & - &\\
		2\pique & - & 3\coeur && \\
		\end{tabular}
		est forcing de manche\\\\
		
		\SSection{Défense contre les Michaëls}
		Le principe général est le suivant :
		\begin{itemize}
		\item Cue bid de la moins chère : La couleure non nommée, FM
		\item Cue bid de la plus chère : Fittée, FM
		\item Autre : Naturel NF\\
		\end{itemize}
		
\Chap{Ouverture mineure}
	\Section{Soutien mineur inversé}
		\SSection{Principe}
		\begin{tabular}{cccc|l}
		1\trefle & - & 2\trefle && Soutien FM (Possiblement 4\coeur/4\pique)\\	
		&& 2\carreau && Soutien limite (9-11HL)\\
		&& 3\trefle && Soutien barrage (6-8 HL)\\
		\end{tabular}\\
		Après passe, le SMI n'est plus FM et dénie 4\coeur/4\pique\\
	
		\begin{tabular}{cccc|l}
		1\trefle & - & 2\trefle & - &\\
		2\carreau/2\coeur/2\pique &&&& Naturel irrégulier (dès 12 points)\\
		2SA &&&& 12-14 OU 18-19 ; Régulier\\
		3\trefle &&&& Unicolore 6$^{ème}$ (Ou 5$^{ème}$ avec un petit doubleton majeur)\\
		3\carreau/3\coeur/3\pique &&&& Bel unicolore 6$^{ème}$ ; Splinter\\
		3SA &&&& 15-17 ; 4441 Singleton \carreau\\
		4\trefle &&&& Chelemisant \trefle\\
		\end{tabular}\\\\

		Même principe sur l'ouverture d'1\carreau\\

		\SSection{Développement sur 2SA}
		\begin{tabular}{cccc|l}
		1\trefle/\carreau & - & 2\trefle/\carreau & - &\\
		2SA & - & 3\trefle && Relais pour 3\carreau\\
		&& 3\carreau/3\coeur && 4\coeur/4\pique\\
		&& 3\pique && Bicolore 5\trefle/5\carreau\\
		&& 3SA && Pour jouer\\
		&& 4\trefle/\carreau && Chelemisant\\
		\end{tabular}\\\\

		\begin{tabular}{cccc|l}
		1\trefle/\carreau & - & 2\trefle/\carreau & - &\\
		2SA & - & 3\trefle & - &\\
		3\carreau & - & 3\coeur/\pique && Singleton \coeur/\pique\\	
		&& 3SA && Singleton \carreau/\trefle\ (Autre mineur)\\
		&& 4\trefle/\carreau && Singleton \carreau/\trefle\ (Autre mineur) ; Chelemisant\\
		\end{tabular}\\\\

		\begin{tabular}{cccc|l}
		1\trefle/\carreau & - & 2\trefle/\carreau & - &\\
		2SA & - & 3\carreau & - &\\
		3\coeur &&&& 18-19 ; Fitté\\
		3SA &&&& 12-14 ; Non fitté\\
		4\coeur &&&& 12-14 ; Fitté\\
		4SA &&&& 18-19 ; Non fitté\\
		\end{tabular}\\
		De même pour les \pique\\

		\SSection{Développement sur 3SA}
		\begin{tabular}{cccc|l}
		1\trefle/\carreau & - & 2\trefle/\carreau & - &\\
		3SA & - & 4\trefle && Chelemisant \trefle/\carreau\\
		&& 4\carreau/4\coeur && 4\coeur/4\pique\ (Le fit est certain)\\
		&& 4SA && Quantitatif\\
		\end{tabular}\\\\

		\SSection{Réactions en cas d'interventions}
		Après un contre ou une intervention, on ne joue plus le soutien mineur inversé\\
		Après un contre, le XX comprend les mains fittées en mineur FM et 2SA est Truscott\\
		Sur une intervention, le Cue-Bid comprend les mains fittées en mineur limite ou FM\\

\newpage
		\SSection{Séquence 1\carreau\ - 2\trefle}
		\begin{tabular}{cccc|l}
		1\carreau & - & 2\trefle & - &\\
		2\carreau &&&& Unicolore 6$^{ème}$ OU Bicolore de première zone\\
		2\coeur/2\pique &&&& Naturel irrégulier (dès 15 points) \\
		2SA &&&& 12-14 OU 18-19 ; Régulier\\
		3\trefle &&&& Naturel irrégulier (dès 15 points)\\
		3SA &&&& 15-17 ; 4441 Singleton \trefle\ (Puis tout Texas)\\
		\end{tabular}\\\\

		\begin{tabular}{cccc|l}
		1\carreau & - & 2\trefle & - &\\
		2\carreau & - & 2\coeur/2\pique && 4\coeur/\pique\ OU Un arrêt pour 3SA\\
		&& 2SA/3\trefle && Naturel limite\\
		&& 3\carreau && Chelemisant \carreau\\
		&& 3\coeur/3\pique && Fit \carreau\ ; Splinter \coeur/\pique\\
		&& 3SA && Naturel\\
		\end{tabular}\\\\

		\begin{tabular}{cccc|l}
		1\carreau & - & 2\trefle & - &\\
		2SA & - & 3\trefle && Relais pour 3\carreau\ (Puis même développement que le SMI)\\
		&& 3\carreau/3\coeur && 4\coeur/4\pique\ (Puis même développement que le SMI)\\
		&& 3\pique && Proposition de chelem à \trefle\ (3SA décourageant)\\
		&& 3SA && Pour jouer\\
		&& 4\trefle/4\carreau && Naturel chelemisant\\
		\end{tabular}\\\\

	\Section{Autres conventions}
		\SSection{Check back stayman}
		Le check back stayman se fait toujours à 3\trefle.\\

		\SSection{Convention Bessis}
		\begin{tabular}{cccc|l}
		1\trefle/\carreau & - & 2\coeur && 6-10 ; 5\pique/4\coeur\ (Possiblement 5/5)\\
		\end{tabular}\\
		ATTENTION : Par inférence, 1\trefle/\carreau\ - 1\pique\ - 1SA - 2\coeur\ promet un 5/4 dans une main limite\\
		De plus, 1\trefle/\carreau\ - 1\pique\ - 1SA - 3\coeur\ promet un 5/5 dans une main limite\\
	
		\begin{tabular}{cccc|l}
		1\trefle/\carreau & - & 2\coeur & - &\\
		2SA & - & 3\trefle && 5/4 minimum\\
		&& 3\carreau && 5/5 minimum\\
		&& 3\coeur && 5/4 maximum\\
		&& 3\pique && 5/5 maximum\\
		\end{tabular}\\\\
	
		La séquence 1\trefle/\carreau\ - 2\pique\ promet par contre 6\pique\ dans une main faible\\
		
		\SSection{En cas d'intervention}
		En cas d'intervention par contre, les développements sont inchangés.\\
		
		On joue rodrigue dans les séquences 1\trefle/\carreau\ 1\pique\ et 1\carreau\ 2\trefle.\\
		
		\SSection{1\trefle\ 2\trefle\ \& 1\trefle\ 2\carreau}
		On joue les michaëls précisés sauf :\\
		
		\begin{tabular}{cccc|l}
		1\trefle & - & 2\trefle && Bicolore majeur\\
		&& 2\carreau && Barrage naturel\\
		\end{tabular}\\\\

\Chap{Intervention}
	\Section{Sur 1SA Fort}
	\underline{En 2$^{ème}$ position}
	
	\begin{tabular}{cccc|l}
	1SA & X &&& Mineur cinquième et majeure quatrième\\
	& 2\trefle &&& Landy 5/4 majeur (2\carreau\ demande de la plus longue)\\
	& 2\carreau &&& Multi\\
	& 2\coeur/2\pique &&& 5\coeur/\pique\ et une mineure 4$^{ème}$ (2SA demande de la mineure)\\
	& 2SA &&& Bicolore 5\trefle/5\carreau\\
	& 3\trefle/3\carreau &&& Naturel\\
	\end{tabular}\\
	Sur le X, 2\trefle\ demande la mineure et 2\carreau\ demande la majeure\\
	
	\underline{En 4$^{ème}$ position}\\
	Tout naturel et X pour les majeurs\\

	\underline{Après un Texas}
	
	\begin{itemize}
	\item X = D'entame (Le contre est d'appel en réveil après la rectification)
	\item Rectification = Bicolore 5/4 majeur/mineur
	\item 2SA = Bicolore 5/5 mineur (De même en réveil après la rectification)
	\item Autre = Unicolore naturel (De même en réveil après la rectification)\\
	\end{itemize}

	\Section{Sur 1SA Faible}
	\begin{tabular}{cccc|l}
	1SA & X &&& Jeu régulier ou Unicolore puissant\\
	& 2\trefle &&& Landy\\
	& 2\carreau/2\coeur/2\pique/3\trefle &&& Texas\\
	& 2SA &&& Bicolore mineur\\
	\end{tabular}\\
	Le X promet 1 point de plus que le maximum de la zone du SA faible (15 H vs 12-14 ; 13 H vs 10-12)\\
	Sur le Texas, la rectification indique un jeu faible et ne promet pas le fit\\

	\begin{tabular}{cccc|l}
	1SA & X & - & Passe & Régulier sans majeure 5$^{ème}$\\
	&&& 2\trefle/2\carreau/2\coeur/2\pique & 0-8 ; Naturel\\
	&&& 2SA/3\trefle/3\carreau/3\coeur & 9+ ; Texas\\
	\end{tabular}\\\\

	\begin{tabular}{cccc|l}
	1SA & 2\trefle & - & 2\carreau & Relais\\
	&&& 2\coeur/2\pique & Préférence\\
	&&& 2SA & Cue-bid\\
	&&& 3\coeur/3\pique & Proposition\\
	&&& 4\coeur/4\pique & Pour les jouer\\
	\end{tabular}\\\\

	\Section{Sur 2\carreau\ Multi}
	\begin{tabular}{cccc|l}
	2\carreau & X &&& Contre d'appel court à \pique\\
	& 2\coeur &&& Contre d'appel court à \coeur\\
	& 2\pique/2SA/3\trefle/3\carreau &&& Naturel\\
	& 3\coeur/3\pique &&& Naturel 6+\coeur/\pique\\
	& 3SA &&& Tendance Gambling\\
	\end{tabular}\\\\
	
	\Section{Sur 1\trefle\ Fort}
	\begin{tabular}{cccc|l}
	1\trefle & -  & X && Bicolore de même Couleure\\
	&& 1\carreau && Bicolore de même Rang\\
	& 1SA &&& Bicolore Mélangé\\
	& 2SA &&& Mineur sixième et majeur quatrième\\
	\end{tabular}\\
	
\Chap{Contres}
	\Section{Contres compétitifs}
		\SSection{Barrage}
		4\coeur\ - X : Appel
		
		4\pique\ - X : Optionnel
		
		5\trefle\ - X : Optionnel (dégagement très rare)
		
		Passe - Passe - 5\trefle\ - X : Punitif
		
		3\trefle\ - Passe - 5\trefle\ - X : Optionnel
		
		Passe - Passe - 3\trefle\ - X - 5\trefle\ - Passe : Forcing\\
		
		1\pique\ - X - 3\pique\ - X : Appel
		
		1\pique\ - X - 3\pique\ - X - Passe - 4\trefle/\carreau\ - Passe - 4\coeur\ : 4\coeur\ et une mineure
		
		2\pique\ - X - 3\pique\ - X : Appel\\
		
		\SSection{Contre de l'ouvreur}
		1\carreau\ - Passe - 1\coeur\ - 2\trefle\ - X : 3\coeur\ et du jeu (Passe ne dénie pas les 3 cartes)
		
		1\pique\ - Passe - 1SA - 2\trefle\ - X : Appel
		
		1\pique\ - Passe - 2\trefle\ - 2\coeur\ - X : Appel avec une main intéressante
		
		1\carreau\ - Passe - 1SA - 2\pique\ - X : Appel
		
		1\coeur\ - Passe - 4\coeur\ - 4\pique\ - X : Punitif (forcing Passe)\\
		
		\SSection{Après action du partenaire => Punitif si deux couleurs nommées}
		1\pique\ - Passe - Passe - X - 2\pique\ - X : Appel
		
		1\pique\ - Passe - Passe - X - 2\trefle\ - X : Punitif
		
		1\pique\ - Passe - Passe - 2\coeur\ - 2\pique\ - X : Appel
		
		1\pique\ - Passe - Passe - 2\carreau\ - 2\coeur\ - X : Punitif\\
		
		1\pique\ - Passe - 1SA - 2\coeur\ - 2\pique\ - X : Appel
		
		1\pique\ - Passe - 1SA - 2\coeur\ - 3\carreau\ - X : Punitif
		
		1\carreau\ - Passe - 1\coeur\ - X - 2\carreau\ - X : Punitif
		
		1\carreau\ - Passe - 1SA - X - 2\carreau\ - X : Appel\\
		
		1\coeur\ - Passe - 2\coeur\ - 3\carreau\ - 3\coeur\ - X : Appel\\
		
		\SSection{Autres}
		1\coeur\ - Passe - 1\pique\ - Passe - 2\coeur\ - X : Appel
		
		1\coeur\ - Passe - 1SA - Passe - 2\coeur\ - X : Punitif
		
		1\coeur\ - Passe - 1SA - Passe - 2\trefle\ - Passe - 2\coeur\ - Passe - Passe - X : Punitif\\
		
		1\carreau\ - Passe - 1\coeur\ - 2\coeur\ - X : Appel
		
		1\carreau\ - Passe - 1\coeur\ - 2\coeur\ - Passe - Passe - X : Appel\\
		
	\Section{Après 1SA}
		1SA - MULTI - 2\coeur\ - X : Punitif (car couleur devinée)
		
		1SA - LANDY - 2\carreau\ - X : Appel
		
		1SA - 2\coeur\ - 2\pique\ - X : Punitif
		
		1SA - minMAJ - 2\trefle\ - X : Appel (Si 2\trefle\ est naturel, cela devient Passe ou corrige)
		
		1SA - Passe - 2\carreau\ - Passe - 2\coeur\ - 2\pique - X : Appel\\
		
		
		
		1\coeur\ - 1SA - Passe - Passe - X : Du jeu et 6\coeur/4\pique
		
		1\trefle\ - 1SA - Passe - Passe - X : Appel court à \carreau\ (relais à 2\carreau\ nomme ta majeur)
\end{document}