\documentclass[a4paper, oneside, 11pt]{report}
\usepackage{Bridge}

\begin{document}
\Chap{Principes généraux}
	\Section{Ouvertures}
		\SSection{1\trefle/1\carreau}
            \begin{itemize}
                \item Meileure mineure  si 12--16 / Carreau 5$^{ème}$ si 17+
                \item Texas sur 1\trefle
                \item Double-deux
                \item 3$^{ème}$/4$^{ème}$ forcing
                \item Pas de rodrigue
                \item Soutien mineur inversé avec majeure quatrième\\
            \end{itemize}
			
		\SSection{1\coeur/1\pique}
            \begin{itemize}
                \item Majeur 5$^{ème}$
                \item 2/1 Forcing de manche
                \item Drury
                \item Rebid 2SA (2\pique\ si ouverture d'1\coeur) FM\\
            \end{itemize}
			
		\SSection{1SA 14--16}
            \begin{itemize}
                \item Majeur 5$^{ème}$ possible
                \item Mineure 6$^{ème}$ possible
                \item Rubensohl\\
            \end{itemize}

        \SSection{Palier de 2}
            \begin{itemize}
                \item 2\trefle\ Bivalent
                \item 2\carreau\ Multi faible
                \item 2\coeur\ Muiderberg
                \item 2\pique\ Muiderberg
                \item 2SA 20--22\\
            \end{itemize}

        \SSection{Barrage}
            \begin{itemize}
                \item Destructif vert
                \item Constructif rouge
                \item 6 cartes possible au palier de 3
                \item 7 cartes possible au palier de 4\\
            \end{itemize}

        \SSection{Autres}
            \begin{itemize}
                \item Blackwood 5 clés, 41/30 (Sauf exclusion 0/1/2/3)
                \item Michaël précisé (1\carreau\ 3\trefle\ et 1\trefle\ 3\trefle\ naturels) mais 1\trefle\ 2\carreau\ barrage (1\trefle\ 2\trefle\ est bicolore)\\
            \end{itemize}

        \SSection{Jeu de la carte}
            \begin{itemize}
                \item Gros appel partout
                \item 4$^{ème}$ meilleure à SA
                \item Pair/impair à la couleure
                \item Défausse pair/impair
            \end{itemize}

\Chap{Ouverture mineure}
    \Section{Texas sur 1\trefle}
        \SSection{Réponses}
            \begin{tabular}{cccc|l}
                1\trefle & - & 1\carreau && Texas \coeur\\
                && 1\coeur && Texas \pique\\
                && 1\pique && Texas SA (Main à base de \carreau)\\
                && 1SA && Naturel
            \end{tabular}\\\\

        \SSection{Développements après 1\carreau}
            \begin{tabular}{cccc|l}
                1\trefle & - & 1\carreau & - &\\
                1\coeur &&&& Régulier 12--13 (4\pique\ possible)\\
                1\pique &&&& Irrégulier 4\pique\\
                1SA &&&& Régulier 17--19 (Puis Stayman/Texas)\\
                2\trefle/2\carreau &&&& Naturel\\
                2\coeur &&&& 4\coeur\ 14--16\\
                2\pique &&&& Bicolore cher\\
                3\carreau/3\coeur &&&& 17--19 ; Splinter (3\coeur\ est splinter \pique)\\
                2SA &&&& Régulier 6\trefle\ 15--16\\
                3\trefle &&&& Irrégulier 6\trefle\ 17+\\
                3\coeur &&&& 4\coeur\ 17--19\\
            \end{tabular}\\\\

            \begin{tabular}{cccc|l}
                1\trefle & - & 1\carreau & - &\\
                1\coeur & - & 1\pique && PAS 4\pique\\
                && Autre && AVEC 4\pique\\
            \end{tabular}\\\\

            \begin{tabular}{cccc|l}
                1\trefle & - & 1\carreau & - &\\
                1\coeur & - & 2\trefle/2\carreau & - & Double deux avec 4\pique\ (sur 2\trefle, 2\carreau\ dénie 4\coeur\ ou 4\pique)\\
                && Autres && Classique avec 4\pique\\
            \end{tabular}\\\\

            \begin{tabular}{cccc|l}
                1\trefle & - & 1\carreau & - &\\
                1\coeur & - & 1\pique & - &\\
                1SA &&&& Régulier 2--3\coeur\\
                2\coeur &&&& Fitté\\
            \end{tabular}\\\\

            \begin{tabular}{cccc|l}
                1\trefle & - & 1\carreau & - &\\
                1\coeur & - & 1\pique & - &\\
                1SA & - & 2\trefle/2\carreau && Double deux\\
                && Autres && Classique\\
            \end{tabular}\\\\

        \SSection{Développements après 1\coeur}
            \begin{tabular}{cccc|l}
                1\trefle & - & 1\coeur & - &\\
                1\pique &&&& Régulier 12--13\\
                1SA &&&& Régulier 17--19 (Puis Stayman/Texas)\\
                2\trefle/2\carreau/2\coeur &&&& Classique\\
                2\pique &&&& 4\pique\ 14--16\\
                2SA &&&& Régulier 6\trefle\ 15--16\\
                3\trefle &&&& Irrégulier 6\trefle\ 17+\\
                3\carreau/3\coeur &&&& Splinter\\
                3\pique &&&& 4\pique\ 17--19\\
            \end{tabular}\\
            Même principe pour les réponses\\

        \SSection{Développements après 1\pique}
            \begin{tabular}{cccc|l}
                1\trefle & - & 1\pique & - &\\
                1SA &&&& Régulier 12--13\\
                2\trefle/2\coeur/2\pique/3\trefle &&&& Classique\\
                2\carreau &&&& Fit 12--13\\
                2SA &&&& Régulier 17--19\\
                3\carreau &&&& Fit 14--16\\
                3\coeur/3\pique &&&& Splinter 17--19\\
            \end{tabular}\\\\

            \begin{tabular}{cccc|l}
                1\trefle & - & 1\pique & - &\\
                1SA & - & 2\trefle/2\carreau && Double deux\\
                && 2\coeur/\pique && 4\coeur/\pique\ FM\\
            \end{tabular}\\\\

            \begin{tabular}{cccc|l}
                1\trefle & - & 1\pique & - &\\
                1SA & - & 2\trefle & - &\\
                2\carreau & - & 2\coeur/\pique && Force limite\\
                && 3\trefle && 5\trefle/5\carreau\ faible\\
            \end{tabular}\\\\

    \Section{Convention sur 1\carreau}
        \SSection{Séquence 1\carreau\ - 2\trefle}
            \begin{tabular}{cccc|l}
                1\carreau & - & 2\trefle & - &\\
                2\carreau &&&& Unicolore 6$^{ème}$ OU Bicolore 12--13\\
                2\coeur/2\pique &&&& Naturel irrégulier (14+) \\
                2SA &&&& Régulier 12--13 OU 17--19\\
                3\trefle &&&& Naturel irrégulier (14+)\\
                3SA &&&& 14--16 ; 4441 Singleton \trefle\ (Puis tout Texas)\\
            \end{tabular}\\\\

            \begin{tabular}{cccc|l}
                1\carreau & - & 2\trefle & - &\\
                2\carreau & - & 2\coeur/2\pique && 4\coeur/\pique\ OU Un arrêt pour 3SA\\
                && 2SA/3\trefle && Naturel limite\\
                && 3\carreau && Chelemisant \carreau\\
                && 3\coeur/3\pique && Fit \carreau\ ; Splinter \coeur/\pique\\
                && 3SA && Naturel\\
            \end{tabular}\\\\

            \begin{tabular}{cccc|l}
                1\carreau & - & 2\trefle & - &\\
                2SA & - & 3\trefle && Relais pour 3\carreau\ (Puis même développement que le SMI)\\
                && 3\carreau/3\coeur && 4\coeur/4\pique\ (Puis même développement que le SMI)\\
                && 3\pique && Proposition de chelem à \trefle\ (3SA décourageant)\\
                && 3SA && Pour jouer\\
                && 4\trefle/4\carreau && Naturel chelemisant\\
            \end{tabular}\\\\

        \SSection{Après rebid à 2SA}
            \begin{tabular}{cccc|l}
                1\carreau & - & 1\pique & - &\\
                2SA & - & 3\trefle && Texas \carreau\ (Fort ou faible)\\
                && 3\carreau && Texas \coeur\ donc 5\pique/4\coeur\ (Rectification fittée)\\
                && 3\coeur && Texas \pique\ donc 5\pique\ (Rectification fittée)\\
                && 3\pique && 6\pique\ chelemisant\\
                && 3SA && Pour jouer\\
            \end{tabular}\\\\

            \begin{tabular}{cccc|l}
                1\carreau & - & 1\pique & - &\\
                2SA & - & 3\trefle & - &\\
                3\carreau & - & 3\coeur/3SA && Courte \coeur/\trefle\\
                && 3\pique && Pour jouer\\
            \end{tabular}\\

    \Section{Soutien mineur inversé}
        \SSection{Réponses}
            \begin{tabular}{cccc|l}
                1\trefle & - & 2\trefle && Soutien FM (Possiblement 4\coeur/4\pique)\\
                && 2\carreau && Soutien limite\\
                && 3\trefle && Soutien barrage\\
            \end{tabular}\\
            Après passe, le SMI n'est plus FM et dénie 4\coeur/4\pique\\

            \begin{tabular}{cccc|l}
                1\trefle & - & 2\trefle & - &\\
                2\carreau/2\coeur/2\pique &&&& Naturel irrégulier\\
                2SA &&&& Régulier 12--13 OU 17--19\\
                3\trefle &&&& Unicolore 6$^{ème}$ (Ou 5$^{ème}$ avec un petit doubleton majeur)\\
                3\carreau/3\coeur/3\pique &&&& Bel unicolore 6$^{ème}$ ; Splinter\\
                3SA &&&& 14--16 ; 4441 Singleton \carreau\\
                4\trefle &&&& Chelemisant \trefle\\
            \end{tabular}\\
            Même principe sur l'ouverture d'1\carreau\\

        \SSection{Développement sur 2SA}
            \begin{tabular}{cccc|l}
                1\trefle/\carreau & - & 2\trefle/\carreau & - &\\
                2SA & - & 3\trefle && Relais pour 3\carreau\\
                && 3\carreau/\coeur && 4\coeur/\pique\\
                && 3\pique && Bicolore 5\trefle/5\carreau\\
                && 3SA && Pour jouer\\
                && 4\trefle/\carreau && Chelemisant\\
            \end{tabular}\\\\

            \begin{tabular}{cccc|l}
                1\trefle/\carreau & - & 2\trefle/\carreau & - &\\
                2SA & - & 3\trefle & - &\\
                3\carreau & - & 3\coeur/\pique && Singleton \coeur/\pique\\
                && 3SA && Singleton \carreau/\trefle\ (Autre mineure)\\
                && 4\trefle/\carreau && Singleton \carreau/\trefle\ (Autre mineure) ; Chelemisant\\
            \end{tabular}\\\\

            \begin{tabular}{cccc|l}
                1\trefle/\carreau & - & 2\trefle/\carreau & - &\\
                2SA & - & 3\carreau & - &\\
                3\coeur &&&& 17--19 ; Fitté\\
                3SA &&&& 12--13 ; Non fitté\\
                4\coeur &&&& 12--13 ; Fitté\\
                4SA &&&& 17--19 ; Non fitté\\
            \end{tabular}\\
            De même pour les \pique\\

        \SSection{Développement sur le rebid à 3SA}
            \begin{tabular}{cccc|l}
                1\trefle/\carreau & - & 2\trefle/\carreau & - &\\
                3SA & - & 4\trefle && Chelemisant \trefle/\carreau\\
                && 4\carreau/4\coeur && 4\coeur/4\pique\ (Le fit est certain)\\
                && 4SA && Quantitatif\\\\
            \end{tabular}\\

    \Section{Autres réponses}
        \begin{tabular}{cccc|l}
            1\trefle/1\carreau & - & 2\coeur && 5\pique/4\coeur\ faible (Relais à 2SA)\\
            && 2\pique && Barrage\\
        \end{tabular}\\\\

        \begin{tabular}{cccc|l}
            1\trefle & - & 2\coeur & - &\\
            2SA & - & 3\trefle && Minimum 5\coeur/4\pique\\
            && 3\carreau && Minimum 5\coeur/5\pique\\
            && 3\coeur && Maximum 5\coeur/4\pique\\
            && 3\pique && Maximum 5\coeur/5\pique\\
        \end{tabular}\\\\

    \Section{Double 2}
        \SSection{Après 2\trefle\ (Limite ou faible)}
            \begin{tabular}{cccc|l}
                1\carreau & - & 1\pique & - &\\
                1SA & - & 2\trefle & - &\\
                2\carreau & - & Passe && Pour jouer\\
                && 2\coeur && Limite 5\pique/4\coeur\\
                && 2\pique && Limite 5\pique\\
                && 2SA && Limite avec du \trefle\\
                && 3\trefle && Faible avec du \trefle\\
                && 3\carreau && Limite avec du \carreau\\
                && 3\coeur && Limite 6\pique/4\coeur\\
                && 3\pique && Limite 6\pique\\
                && 3SA && Proposition entre 3SA et 4\pique\\
            \end{tabular}\\
            Même principe sur les autres séquences\\

        \SSection{Après 2\carreau\ (Forcing manche)}
                \begin{tabular}{cccc|l}
                    1\carreau & - & 1\pique & - &\\
                    1SA & - & 2\carreau & - &\\
                    2\coeur &&&& 4\coeur\ (Ne dénie pas 3\pique)\\
                    2\pique &&&& 3\pique\\
                    2SA &&&& Arrêt \coeur\ et \carreau/\trefle\\
                    3\trefle &&&& Arrêt pour SA\\
                    3\carreau &&&& Belle couleure, mauvais arrêts\\
                \end{tabular}\\
                Même principe sur les autres séquences\\

    \Section{Troisième forcing}
        \SSection{Réponses}
            \begin{tabular}{cccc|l}
                1\carreau & - & 1\pique & - &\\
                2\carreau & - & 2\coeur & - & Troisième forcing\\
                2\pique &&&& 12--13 ; 3\pique\\
                2SA &&&& Minimum ; Tous les arrêts\\
                3\trefle &&&& Minimum\\
                3\carreau &&&& 14--16 ; Un arrêt pour 3SA\\
                3\coeur &&&& 14--16 ; 4\coeur\\
                3\pique &&&& 14--16 ; 3\pique\\
                3SA &&&& 14+ ; Tous les arrêts\\
            \end{tabular}\\
            Même principe sur les autres séquences\\

        \SSection{Développements particuliers}
            \begin{tabular}{cccc|l}
                1\carreau & - & 1\pique & - &\\
                2\carreau & - & 2\coeur & - & Troisième forcing\\
                2\pique & - & 2SA && Enchère d'essai à \pique\\
                && 3\pique && Chelemisant\\
            \end{tabular}\\
            De même sur les autres séquences\\

        \SSection{Séquence particulière}
            \begin{tabular}{cccc|l}
                1\trefle & - & 1\coeur(\pique) & - &\\
                2\trefle & - & 2\carreau && Troisième forcing sans 4\coeur\\
                && 2\coeur && Troisième forcing avec 4\coeur\\
            \end{tabular}\\\\

    \Section{En cas d'intervention}
        \SSection{Intervention par X}
            \begin{tabular}{cccc|l}
                1\trefle & X & XX && Classique\\
                && 1\carreau/1\coeur/1\pique && Texas\\
                && 1SA && Naturel\\
                && 2\trefle && Naturel NF\\
                && 2\carreau/2\coeur/2\pique && Barrage\\
                && 2SA && Truscott\\
                && 3\trefle && Barrage\\
            \end{tabular}\\\\

            \begin{tabular}{cccc|l}
                1\carreau & X & XX && Du jeu\\
                && 1\coeur/1\pique && Naturel\\
                && 1SA && Naturel\\
                && 2\trefle && Naturel NF\\
                && 2\carreau && Fit\\
                && 2\coeur/2\pique && Barrage\\
                && 2SA && Truscott\\
                && 3\trefle && Barrage\\
                && 3\carreau && Barrage\\
            \end{tabular}\\\\

        \SSection{Intervention à la couleur}
            \begin{tabular}{cccc|l}
                1\trefle & 1\carreau & X && Texas \coeur\\
                && 1\coeur && Texas \pique\\
                && 1\pique && Texas SA\\
                && 1SA/2\carreau/2\coeur\ && Texas (Fort ou faible)\\
                && 2\trefle && Truscott (2\carreau\ est demande d'arrêt)\\
                && 2\pique && Limite + ; Cue-bid\\
                && 2SA && Naturel limite\\
                && 3\trefle && Barrage\\
                && 3\carreau && Barrage\\
            \end{tabular}\\\\

            \begin{tabular}{cccc|l}
                1\carreau & 1\coeur & X && Texas \pique\\
                && 1\pique && Texas SA\\
                && 1SA/2\trefle/2\coeur\ && Texas (Fort ou faible)\\
                && 2\carreau && Truscott (2\coeur\ est demande d'arrêt)\\
                && 2\pique && Limite + ; Cue-bid\\
                && 2SA && Naturel limite\\
                && 3\trefle && Barrage\\
                && 3\carreau && Barrage\\
            \end{tabular}\\\\

        \SSection{Intervention à la couleur en texas}
            \begin{tabular}{cccc|l}
                1\trefle & 1\carreau (\coeur) & X && Du \carreau\\
                && 1\coeur && Texas \pique\\
                && 1\pique && Texas SA\\
                && 1SA/2\trefle && Naturel\\
                && 2\carreau/2\coeur && Texas (Fort ou faible)\\
            \end{tabular}\\\\

            \begin{tabular}{cccc|l}
                1\trefle & 1\pique (\carreau) & X && Appel aux majeures\\
                && 1SA/2\trefle && Naturel\\
                && 2\carreau/2\coeur && Texas (Fort ou faible)\\
            \end{tabular}\\\\

\Chap{Ouverture majeure}
	\Section{Réponses}
        \SSection{Sans passe d'entrée}
            \begin{tabular}{cccc|l}
                1\coeur & - & 1SA && Peut cacher un fit faible\\
                && 2\coeur && 6--10 ; Fitté par 3\\
                && 2\pique && Barrage 6\pique\\
                && 2SA && Fitté limite par 3 ou 4 OU Régulier 12--14 ; Fitté par 3\\
                && 3\trefle/\carreau && 6 cartes limite\\
                && 3\coeur && 6--10 ; Fitté par 4\\
                && 3\pique/4\trefle/4\carreau && Splinter\\
                && 3SA && 12--14 régulier ; Fitté par 4\\
                && 4\coeur && Barrage\\
            \end{tabular}\\\\

            Même principe sur 1\pique, avec l'enchère de 3\coeur\ qui est une main limite, sans le fit \pique, avec 6\coeur\\

	    \SSection{Après passe d'entrée}
            \begin{tabular}{cccc|l}
                & (-) & - & - &\\
                1\coeur & - & 1SA && Peut cacher un fit faible\\
                && 2\trefle && Drury\\
                && 2\coeur && 6--10 ; Fitté par 3\\
                && 2\pique/3\trefle/3\carreau && Rencontre 5\pique/\trefle/\carreau ; Fitté par 4\\
                && 2SA && Fitté limite par 4 ; 4441 (3\trefle\ demande la courte)\\
                && 3\coeur && 6--10 ; Fitté par 4\\
                && 3\pique/4\trefle/4\carreau && Splinter\\
                && 4\coeur && Barrage\\
            \end{tabular}\\\\

            \begin{tabular}{cccc|l}
                & (-) & - & - &\\
                1\coeur & - & 2\trefle & - &\\
                2\carreau &&&& Demande de description\\
                2\coeur &&&& Pour jouer\\
                4\coeur &&&& Pour jouer\\
            \end{tabular}\\\\

            \begin{tabular}{cccc|l}
                & (-) & - & - &\\
                1\coeur & - & 2\trefle & - &\\
                2\carreau & - & 2\coeur && Fitté par 3\\
                && 2\pique/3\trefle/3\carreau && Rencontre 5\pique/\trefle/\carreau ; Fitté par 3\\
                && 3\coeur && Fitté par 4\\
            \end{tabular}\\\\

            Même principe pour l'ouverture d'1\pique\ avec 1\pique\ 2\trefle\ 2\coeur\ qui promet un jeu sans l'ouverture dans un 5/4\\

        \SSection{En cas d'intervention par X}
            \begin{itemize}
                \item 1\coeur\ X 2\carreau\ est un fit positif (2\coeur\ est barrage)
                \item Les changements de couleur à saut sont naturels faibles non fitté\\
            \end{itemize}

        \SSection{En cas d'intervention à la couleur}
            \begin{itemize}
                \item 3SA est naturel
                \item Le Cue-bid est un fit FM
                \item Les changements de couleur à saut sont naturels faibles non fitté\\
            \end{itemize}

	\Section{Séquences particulières}
		\SSection{2/1 forcing de manche}
            \begin{itemize}
                \item 1\pique\ 2\carreau\ 2\pique\ 2SA est un relais naturel FM promettant 2\pique
                \item 1\pique\ 2\carreau\ 2\pique\ 3SA est pour les jouer et dénie 2\pique
                \item 1\pique\ 2\carreau\ 2\coeur\ 2\pique\ est chelemisant fitté par 3
                \item 1\pique\ 2\carreau\ 2\coeur\ 3\pique\ est chelemisant fitté par 4\\
            \end{itemize}

            \begin{itemize}
                \item 1\pique\ 2\carreau\ 3\trefle/3\carreau\ promet 14+
                \item 1\pique\ 2\carreau\ 2\pique\ 2SA 3\trefle/3\carreau\ promet désormais 12--13\\
            \end{itemize}

        \SSection{1\coeur\ 1SA 2\pique\ \& 1\pique\ 1SA 2SA FM}
            \begin{tabular}{cccc|l}
                1\coeur & - & 1SA & - &\\
                2\pique & - & 2SA && Relais\\
                && 3\coeur && Fit faible\\
            \end{tabular}\\
            Le bicolore 5\coeur/4\pique\ est alors décrit par la séquence 1\coeur\ 1SA 2SA\\

            \begin{tabular}{cccc|l}
                1\coeur & - & 1SA & - &\\
                2\pique & - & 2SA & - &\\
                3\trefle/3\carreau &&&& Bicolore 54\\
                3\coeur &&&& 6\coeur\ moches\\
                3SA &&&& 5332 ; Petit doubleton \pique\\
            \end{tabular}\\
            Par inférence 1\coeur\ 1SA 3\trefle/3\carreau\ promet un 55\\

            \begin{tabular}{cccc|l}
                1\pique & - & 1SA & - &\\
                2SA & - & 3\trefle & - &\\
                3\carreau/3\coeur &&&& Bicolore 54\\
                3\pique &&&& 6\pique\ moches\\
                3SA &&&& 5332 ; 3\coeur\\
            \end{tabular}\\
            Par inférence 1\pique\ 1SA 3\carreau/3\coeur\ promet un 55 mais pas 3\trefle\ (Avec un 55 noir, on ouvre d'1\trefle)\\

        \SSection{1\coeur\ 1\pique\ 2SA}
            \begin{tabular}{cccc|l}
                1\coeur & - & 1\pique & - &\\
                2SA &&&& 17--19 ; Régulier\\
            \end{tabular}\\\\

            \begin{tabular}{cccc|l}
                1\coeur & - & 1\pique & - &\\
                2SA & - & 3\carreau && Texas \coeur\ (Fit faible ou fort) OU 5\pique\ faible\\
                && 3\coeur && Texas \pique\ (Rectification fittée)\\
                && 3\pique && 6\pique\ chelemisant\\
                && 3SA &&\\
            \end{tabular}\\\\

            \begin{tabular}{cccc|l}
                1\coeur & - & 1\pique & - &\\
                2SA & - & 3\carreau & - &\\
                3\coeur & - & 3\pique && Pour jouer\\
                && 3SA && Controle \pique\ chelemisant \coeur\\
            \end{tabular}\\\\

        \SSection{Double deux}
            Même développements que sur ouverture mineure\\

		\SSection{Défense contre les Michaëls}
            Le principe général est le suivant :
            \begin{itemize}
                \item Cue bid de la moins chère : La couleure non nommée, FM
                \item Cue bid de la plus chère : Fittée, FM
                \item Autre : Naturel NF\\
            \end{itemize}

\Chap{Ouverture de 1SA}
    \Section{Réponses}
        \begin{tabular}{cccc|l}
            1SA & - & 2\trefle && Stayman 3 paliers\\
            && 2\carreau/2\coeur/2\pique/3\trefle && Texas\\
            && 2SA && 9HL\\
            && 3\carreau && Stayman 4333\\
            && 3\coeur/\pique && 6\coeur/\pique\ chelemisant\\
            && 3SA && 10--16HL\\
            && 4\trefle && Bicolore majeur au moins 6/5\\
            && 4\carreau && Bicolore majeur 5/5\\
            && 4\coeur/4\pique && Naturel\\
            && 4SA && 17--18HL Quantitatif\\
        \end{tabular}\\\\

	\Section{Développements}
        \begin{tabular}{cccc|l}
            1SA & - & 2\trefle & - &\\
            2\carreau & - & 2\coeur && 4\coeur/4\pique\ faible\\
        \end{tabular}\\\\

        \begin{tabular}{cccc|l}
            1SA & - & 2\carreau & - &\\
            2\coeur & - & 2\pique && Relais FM\\
            && 2SA && Limite 5\coeur\\
            && 3\trefle && 5\coeur/5\trefle\\
            && 3\carreau && 5\coeur/5\carreau\\
            && 3\coeur && Limite 6\coeur\\
            && 3\pique/4\trefle/4\carreau && Splinter\\
            && 4\coeur && Pour jouer\\
        \end{tabular}\\
        Même principe pour les \pique\ (2SA est relais FM)\\

        \begin{tabular}{cccc|l}
            1SA & - & 2\carreau & - &\\
            2\coeur & - & 2\pique & - &\\
            2SA & - & 3\trefle && 5\coeur/4\trefle\ ; Résidu mineur\\
            && 3\carreau && 5\coeur/4\trefle\ ; Résidu majeur\\
            && 3\coeur && 5\coeur/4\carreau\ ; Résidu mineur\\
            && 3\pique && 5\coeur/4\carreau\ ; Résidu majeur\\
        \end{tabular}\\
        Même principe pour les \pique\\

        \begin{tabular}{cccc|l}
            1SA & - & 2\pique & - &\\
            3\trefle & - & 3\carreau && 5\trefle/5\carreau\\
            && 3\coeur && Singleton \coeur\\
            && 3\pique && Singleton \pique\\
            && 3SA && Singleton \trefle\\
        \end{tabular}\\\\

        \begin{tabular}{cccc|l}
            1SA & - & 2\pique & - &\\
            2SA & - & 3\trefle && Pour jouer\\
            && 3\carreau && Singleton \carreau\\
            && 3\coeur && Singleton \coeur\\
            && 3\pique && Singleton \pique\\
            && 3SA && Pour jouer (Exemple ADxxxx)\\
        \end{tabular}\\\\

        \begin{tabular}{cccc|l}
            1SA & - & 4\trefle & - &\\
            4\carreau &&&& Pas de préférence\\
            4\coeur/\pique &&&& Pour jouer\\
        \end{tabular}\\\\

	\Section{Réaction face aux interventions}
		\SSection{Contre du Stayman/Texas}
            Si le Texas est contré, on fait jouer de l'autre main :
            \begin{itemize}
                \item Passe = Non fitté
                \item XX = Fitté\\
            \end{itemize}
		
            Si le 2\trefle\ Stayman est contré :\\
            \begin{tabular}{cccc|l}
                1SA & - & 2\trefle & X &\\
                Passe &&&& Pas de majeure\\
                XX &&&& Pour jouer\\
                2\carreau &&&& Naturel\\
                2\coeur/2\pique &&&& 4\coeur/\pique\\
            \end{tabular}\\\\

            \begin{tabular}{cccc|l}
                1SA & - & 2\trefle & X &\\
                - & - & XX && 9+ ; FORCING\\
                && 2\carreau/2\coeur/2\pique && Faible\\
                && 3\trefle && Demande d'arrêt\\
            \end{tabular}\\\\
		
		\SSection{Autres}
            \begin{tabular}{cccc|l}
                1SA & 2\trefle\ (Landy) & X && Punitif dans au moins une majeure OU 8H+ sans arrêt\\
                && 2\carreau && Naturel NF\\
                && 2\coeur/2\pique\ && Singleton \coeur/\pique\ avec un 54 mineur\\
                && 2SA/3\trefle && Texas \trefle/\carreau\\
                && 3\carreau && Une majeure 5$^{ème}$\\
                && 3\coeur/3\pique && Chicane \coeur/\pique\ avec un 55 mineur\\
            \end{tabular}\\\\

    \Section{Cas de l'intervention par 1SA}
        \SSection{Par nous}
            Peu importe la séquence, on oublie les enchères précédentes (2\trefle\ Stayman, etc)\\

            \begin{tabular}{cccc|l}
                1X & 1SA & X & Passe & 4333 \\
                &&& XX & Relais pour Baron (4432 ou 4441)\\
                &&& 2\trefle/2\carreau/2\coeur/2\pique\ & Naturel\\
            \end{tabular}\\\\

            \begin{tabular}{cccc|l}
                1X & 1SA & X & - &\\
                - & XX &&& Relais pour Baron\\
                & 2\trefle/2\carreau/2\coeur/2\pique\ &&& Naturel\\
            \end{tabular}\\\\

        \SSection{Par les adversaires}
            Landyk classique\\

\Chap{Ouverture de 2\trefle/2\carreau/2\coeur/2\pique/2SA et 3SA}
	\Section{2\trefle\ Bivalent}
        \SSection{Réponses}
            \begin{tabular}{cccc|l}
                2\trefle & - & 2\carreau && Relais FM\\
                && 2\coeur && Pour jouer en face d'un 2\coeur\ fort\\
                && 2\pique && Pour jouer en face d'un 2\pique\ fort\\
                && 3\trefle && Jeu faible, au moins 5-4 en majeur (Stayman)\\
                && 3\carreau && Jeu faible, 4-4 en majeur (Stayman)\\
            \end{tabular}\\\\

        \SSection{Développements particuliers}
            \begin{tabular}{cccc|l}
                2\trefle & - & 3\trefle &&\\
                3\carreau &&&& Demande d'une majeur cinquième (réponse en chassé croisé)\\
                3\coeur/3\pique/4\trefle/4\carreau &&&& Naturel chelemisant\\
                3SA &&&& Pour jouer\\
                4\coeur/4\pique/5\trefle/5\carreau &&&& Pour jouer\\
            \end{tabular}\\
            Même principe pour la séquence 2\trefle\ - 3\carreau\\

        \SSection{En cas d'intervention}
            \begin{itemize}
                \item X (ou XX) est négatif
                \item Passe est encourageant\\
            \end{itemize}

	\Section{2\carreau\ Multi}
        \SSection{Réponses}
            \begin{tabular}{cccc|l}
                2\carreau & - & 2\coeur/2\pique/3\coeur/3\pique && Passe ou Corrige\\
                && 2SA && Relais\\
                && 3\trefle && Naturel ; Pour jouer\\
                && 3\carreau && Forcing avec l'autre majeure présumée\\
                && 4\trefle && Nomme ta majeur en Texas\\
                && 4\carreau && Nomme ta majeur\\
                && 4\coeur/4\pique && Naturel\\
            \end{tabular}\\\\

        \SSection{Développements particuliers}
            \begin{tabular}{cccc|l}
                2\carreau & - & 2SA & - &\\
                3\trefle/3\carreau &&&& Minimum \coeur/\pique\ (3\carreau/\coeur\ pour vérifier qu'il s'agit d'un vrai barrage)\\
                3\coeur/3\pique &&&& Maximum \pique/\coeur\\
            \end{tabular}\\\\
	
            \begin{tabular}{cccc|l}
                2\carreau & - & 3\carreau & - &\\
                3\coeur/\pique &&&& Unicolore \coeur/\pique\ sans 3\pique/\coeur\\
                4\trefle &&&& Unicolore \coeur\ avec 3\pique\\
                4\carreau &&&& Unicolore \pique\ avec 3\coeur\\
            \end{tabular}\\\\

        \SSection{En cas d'intervention}
            \begin{tabular}{cccc|l}
                2\carreau & X & Passe && Du carreau ; Accepte 2\carreau\ X\\
                && XX && Pour jouer 2\coeur\ ou 2\pique\\
                && 2\coeur && Pour jouer 2\coeur\ ou 3\pique\\
                && Autre && Classique\\
            \end{tabular}\\\\
            En cas d'intervention naturelle majeure, le contre est passe ou corrige\\
            En cas d'intervention naturelle par 3\trefle, les enchères 4\trefle\ et 4\carreau\ gardent leur sens\\
            En cas d'intervention naturelle par 3\carreau, 4\trefle\ est non forcing et 4\carreau\ demande la majeur\\

\newpage
	\Section{2\coeur\ \& 2\pique\ Muiderberg}
        \SSection{Réponses}
            \begin{tabular}{cccc|l}
                2\pique && 2SA && Relais descriptif\\
                && 3\trefle/4\trefle/5\trefle && Passe ou Corrige\\
                && 3\coeur && 6\coeur\ forcing\\
                && 3\pique/4\pique && Barrage\\
                && 3SA && Pour jouer\\
            \end{tabular}\\\\
	
            Même principe sur 2\coeur\ avec les enchères:
            \begin{tabular}{cccc|l}
                2\coeur && 2\pique && Passe (doubleton) ou Corrige\\
                && 3\pique && 6\pique\ forcing\\
            \end{tabular}\\\\

        \SSection{Développements particuliers}
            \begin{tabular}{cccc|l}
                2\coeur/\pique & - & 2SA & - &\\
                3\trefle/3\carreau &&&& 5\coeur/\pique\ et 4\trefle/4\carreau\ Minimum\\
                3\coeur/3\pique &&&& 5\coeur/\pique\ et 4\trefle/4\carreau\ Maximum\\
                4\trefle/4\carreau &&&& 5\coeur/\pique\ et 5\trefle/5\carreau\ Maximum\\
            \end{tabular}\\\\

            Sur les réponses de 3\trefle/3\carreau/3\coeur/3\pique:
            \begin{itemize}
                \item 3\coeur/\pique\ est pour les jouer
                \item 4\trefle/\carreau\ est chelemisant dans la mineure
                \item Le reste est chelemisant dans la majeure\\
            \end{itemize}

        \SSection{En cas d'intervention}
            Après X ou sur une ouverture en $3^{ème}$, 2SA devient un relais pour la mineur et 3\trefle/\carreau\ sont naturels\\

	\Section{2SA}
        \begin{itemize}
            \item Si le répondant n'a pas passé d'entrée : rectification fittée
            \item Si le répondant a passé d'entrée : rectification obligatoire\\
        \end{itemize}

	\Section{3SA Gambling}
        \SSection{Réponses}
            \begin{tabular}{cccc|l}
                3SA & - & 4\trefle && Passe/Corrige\\
                && 4\carreau && Demande de singleton\\
                && 4\coeur/4\pique/5\trefle/5\carreau/6\trefle/6\carreau && Pour jouer\\
                && 4SA && Nomme ta mineure\\
            \end{tabular}\\\\

        \SSection{Développements particuliers}
            \begin{tabular}{cccc|l}
                3SA & - & 4\carreau & - &\\
                4\coeur &&&& Singleton \coeur\\
                4\pique &&&& Singleton \pique\\
                4SA &&&& 7222\\
                5\trefle &&&& Singleton \carreau\ donc avec les \trefle\\
                5\carreau &&&& Singleton \trefle\ donc avec les \carreau\\
            \end{tabular}\\\\

\Chap{Intervention}
	\Section{Sur 1SA Fort}
        \underline{En 2$^{ème}$ position}

        \begin{tabular}{cccc|l}
            1SA & X &&& Mineur cinquième et majeure quatrième\\
            & 2\trefle &&& Landy 5/4 majeur (2\carreau\ demande de la plus longue)\\
            & 2\carreau &&& Multi\\
            & 2\coeur/2\pique &&& 5\coeur/\pique\ et une mineure 4$^{ème}$ (2SA demande de la mineure)\\
            & 2SA &&& Bicolore 5\trefle/5\carreau\\
            & 3\trefle/3\carreau &&& Naturel\\
        \end{tabular}\\
        Sur le X, 2\trefle\ demande la mineure et 2\carreau\ demande la majeure\\

        \underline{En 4$^{ème}$ position}\\
        Tout naturel et X pour les majeurs (facilement transformable)\\

        \underline{Après un Texas}

        \begin{itemize}
        \item X = D'entame (Le contre est d'appel en réveil après la rectification)
        \item Rectification = Bicolore 5/4 majeur/mineur
        \item 2SA = Bicolore 5/5 mineur (De même en réveil après la rectification)
        \item Autre = Unicolore naturel (De même en réveil après la rectification)\\
        \end{itemize}

	\Section{Sur 1SA Faible}
        \begin{tabular}{cccc|l}
            1SA & X &&& Jeu régulier ou Unicolore puissant\\
            & 2\trefle &&& Landy\\
            & 2\carreau/2\coeur/2\pique/3\trefle &&& Texas\\
            & 2SA &&& Bicolore mineur\\
        \end{tabular}\\
        Le X promet 1 point de plus que le maximum de la zone du SA faible (15 H vs 12--14 ; 13 H vs 10--12)\\
        Sur le Texas, la rectification indique un jeu faible et ne promet pas le fit\\

        \begin{tabular}{cccc|l}
            1SA & X & - & Passe & Régulier sans majeure 5$^{ème}$\\
            &&& 2\trefle/2\carreau/2\coeur/2\pique & 0-8 ; Naturel\\
            &&& 2SA/3\trefle/3\carreau/3\coeur & 9+ ; Texas\\
        \end{tabular}\\\\

        \begin{tabular}{cccc|l}
            1SA & 2\trefle & - & 2\carreau & Relais\\
            &&& 2\coeur/2\pique & Préférence\\
            &&& 2SA & Cue-bid\\
            &&& 3\coeur/3\pique & Proposition\\
            &&& 4\coeur/4\pique & Pour les jouer\\
        \end{tabular}\\\\

	\Section{Sur 2\carreau\ Multi}
        \begin{tabular}{cccc|l}
            2\carreau & X &&& Contre d'appel court à \pique\\
            & 2\coeur &&& Contre d'appel court à \coeur\\
            & 2\pique/2SA/3\trefle/3\carreau &&& Naturel\\
            & 3\coeur/3\pique &&& Naturel 6+\coeur/\pique\\
            & 3SA &&& Tendance Gambling\\
        \end{tabular}\\\\
	
	\Section{Sur 1\trefle\ Fort}
        \begin{tabular}{cccc|l}
            1\trefle & X &&& Bicolore de même Couleure\\
            & 1\carreau &&& Bicolore de même Rang\\
            & 1SA &&& Bicolore Mélangé\\
        \end{tabular}\\
\end{document}
