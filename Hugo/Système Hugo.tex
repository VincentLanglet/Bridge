\documentclass[a4paper, oneside, 11pt]{report}
\usepackage{Bridge}

\begin{document}
\Chap{Ouverture de 1SA}
	\Section{Réponses}
	\begin{tabular}{cccc|l}
	1SA & - & 2\trefle && Stayman 5 paliers\\
	&& 2\carreau/\coeur && Texas \coeur/\pique\\
	&& 2\pique && Texas \trefle\ OU Bicolore 5\trefle/5\carreau\\
	&& 2SA && Texas \carreau\ fort OU 6\carreau\ par 2GH\\
	&& 3\trefle && Texas \carreau\ faible OU Mains diverses\\
	&& 3\carreau && Stayman FM 4333\\
	&& 3\coeur/\pique && Chelemisant \coeur/\pique\\
	&& 3SA && Pour jouer\\
	&& 4\trefle && Bicolore majeur au moins 5/5\\
	&& 4\carreau/\coeur && Texas \coeur/\pique\\
	&& 4SA/5SA && Quantitatif\\
	\end{tabular}\\\\

	\underline{Cas de l'intervention par 1SA :} On ignore l'ouverture
	
	En cas de contre punitif, on joue Tout-Texas :\\
	\begin{tabular}{cccc|l}
	1X & 1SA & X & Passe & Texas XX (On pourra nommer une couleur cinquième)\\
	&&& XX & Texas \trefle\\
	&&& 2\trefle/2\carreau/2\coeur & Texas \carreau/\coeur/\pique\\
	\end{tabular}\\
	Sur le Texas XX, on développe en Baron\\

	En cas de réponse à l'ouverture, on joue Rubensohl :\\
	\begin{tabular}{cccc|l}
	1X & 1SA & 2Y & ... & Rubensohl\\
	\end{tabular}
	
	C'est aussi le cas dans une séquence du type :\\
	\begin{tabular}{cccc|l}
	1X & 1Y & - & - & \\
	1SA & 2Z & ... && Rubensohl\\
	\end{tabular}\\

	En cas de réveil par 1SA, les enchères restent aussi les mêmes (En décalant les zones de points)\\
	
	\underline{Cas de l'intervention adverse par 1SA :} Landik amélioré
	
	\begin{tabular}{cccc|l}
	1\trefle/\carreau & 1SA & 2\trefle && Appel aux majeures \\
	&& 2\carreau/2\coeur/2\pique/3\trefle && Texas\\
	&& 2SA && Bicolore mineur\\
	\end{tabular}\\\\
	
	\begin{tabular}{cccc|l}
	1\coeur/\pique & 1SA & 2\trefle/2\carreau/3\trefle && Texas \\
	&& 2\coeur/\pique && Fit naturel\\
	&& 2\carreau/\coeur && Texas \pique/\trefle\\
	&& 2SA && Bicolore mineur\\
	\end{tabular}\\

\newpage
	\Section{Développements}
		\SSection{Stayman 5 paliers}
		\begin{tabular}{cccc|l}
		1SA & - & 2\trefle & - &\\
		2\carreau & - & 2\coeur/2\pique && 5\coeur/\pique\ limite\\
		&& 3\coeur/3\pique && Chassé-croisé (Sur 3SA on remet 4\coeur/4\pique\ avec un 6/4 fort)\\
		\end{tabular}\\\\

		\begin{tabular}{cccc|l}
		1SA & - & 2\trefle & - &\\
		2SA/3\trefle & - & 3\carreau/3\coeur && Texas\\
		&& 3\pique && Prépare un Splinter fitté \coeur\ (Relais à 3SA)\\
		&& 3SA && Main limite sans majeure (Si déclarant maximum)\\
		&& 4\trefle/4\carreau/4\coeur && Splinter \trefle/\carreau/\coeur\ fitté \pique\\
		\end{tabular}\\
		Sur le relais à 3SA, on nommera sa courte naturellement\\

		\SSection{Texas Majeur}
		\begin{tabular}{cccc|l}
		1SA & - & 2\carreau & - &\\
		2\coeur & - & 2\pique && 5\coeur/5\pique\ faible\\
		&& 2SA && Relais FM 5431 ($\star$)\\
		&& 3\trefle/3\carreau && Naturel dans 5 cartes (Ou 5422 fort)\\
		&& 3\coeur && 6\coeur\ limite\\
		&& 3\pique/4\trefle/4\carreau && 6\coeur\ Splinter\\
		&& 3SA && Passe ou Corrige\\
		\end{tabular}\\\\
		
		\begin{tabular}{cccc|l}
		1SA & - & 2\coeur & - &\\
		2\pique & - & 2SA && Relais FM 5431 ($\star$)\\
		&& 3\trefle/3\carreau && Naturel dans 5 cartes (Ou 5422 fort)\\
		&& 3\coeur && 5\coeur/5\pique\ dans une main de chelem\\
		&& 3\pique && 6\pique\ limite\\
		&& 3SA && Passe ou Corrige\\
		&& 4\trefle/4\carreau/4\coeur && 6\pique\ Splinter\\
		\end{tabular}\\\\
		Sur 3\coeur,  3\pique\ est chelemisant \pique\ et le reste est un contrôle à l'atout \coeur\ (sauf 3SA négatif)\\
	
		\underline{Relais forcing de manche} ($\star$)\\
		Sur l'enchère de 2SA forcing de manche, 3\trefle\ est un relais.
		\begin{itemize}
		\item 3\carreau\ annoncera 4\trefle\ et un résidu mineur
		\item 3\coeur\ annoncera 4\trefle\ et un résidu majeur
		\item 3\pique\ annoncera 4\carreau\ et un résidu mineur
		\item 3SA annoncera 4\carreau\ et un résidu majeur
		\item 4\trefle\ annoncera 4\carreau\ et un résidu majeur (Trop fort pour jouer 3SA)
		\end{itemize}

\newpage
		\SSection{Texas Mineur}
		Sur 2\pique\ (respectivement 2SA), 2SA (3\trefle) est positif et promet au moins un GH 3$^{ème}$\\

		\begin{tabular}{cccc|l}
		1SA & - & 2\pique & - &\\
		2SA & - & 3\trefle && Faible ; Pour jouer\\
		&& 3\carreau && Singleton \carreau\\
		&& 3\coeur/3\pique && Singleton \pique/\coeur\\
		&& 3SA && Pour jouer (Unicolore \trefle\ 6$^{ème}$ par 2GH)\\
		&& 4\trefle && Chelemisant sans singleton\\
		\end{tabular}\\\\

		\begin{tabular}{cccc|l}
		1SA & - & 2\pique & - &\\
		3\trefle & - & 3\carreau && Bicolore 5\trefle/5\carreau\ de manche\\
		&& 3\coeur/3\pique/3SA && Singleton \pique/\coeur/\carreau\\
		&& 4\trefle && Chelemisant sans singleton\\
		&& 4\carreau && Singleton \carreau\ ; Chelemisant\\
		\end{tabular}\\\\
		
		\begin{tabular}{cccc|l}
		1SA & - & 2SA & - &\\
		3\trefle & - &  3\carreau && Singleton \trefle\\
		&& 3\coeur/3\pique && Singleton \pique/\coeur\\
		&& 3SA && Pour jouer (Unicolore \carreau\ 6$^{ème}$ par 2GH)\\
		&& 4\trefle && Singleton \trefle ; Chelemisant\\
		&& 4\carreau && Chelemisant sans singleton\\
		\end{tabular}\\\\

		\begin{tabular}{cccc|l}
		1SA & - & 2SA & - &\\
		3\carreau & - & 3\coeur/3\pique && Singleton \pique/\coeur\\
		&& 3SA && Singleton \trefle\\
		&& 4\trefle && Singleton \trefle ; Chelemisant\\
		&& 4\carreau && Chelemisant sans singleton\\
		\end{tabular}\\\\

		\SSection{Autres}
		\begin{tabular}{cccc|l}
		1SA & - & 3\trefle & - &\\
		3\carreau & - & Passe && Texas \carreau\ faible\\
		&& 3\coeur && 5/4 mineur ; Singleton \pique\\
		&& 3\pique && 5/4 mineur ; Singleton \coeur\\
		&& 3SA && 5/4 mineur sans singleton ; Espoir de chelem\\
		&& 4\trefle/4\carreau && 5/4 mineur sans singleton ; Chelemisant (On nomme la $5^{ème}$)\\
		\end{tabular}\\\\
	
		\begin{tabular}{cccc|l}
		1SA & - & 3\carreau & - & \it{Stayman 4333}\\
		3\coeur &&&& 4\coeur\ et intérêt pour l'atout (Et peut être 4\pique)\\
		3\pique &&&& 4\pique\ et intérêt pour l'atout\\
		3SA &&&& Pas de majeure OU Pas d'intérêt pour l'atout\\
		\end{tabular}\\\\
	
		\begin{tabular}{cccc|l}
		1SA & - & 4\trefle & - & \it{Bicolore majeur au moins 5/5}\\
		4\carreau &&&& Relais pour connaître la majeure la plus longue\\
		4\coeur/4\pique &&&& Pour jouer\\
		\end{tabular}\\\\

\newpage
	\Section{Réaction face aux interventions}
		\SSection{Contre du Stayman/Texas}
		Si le Texas est contré :
		\begin{itemize}
		\item Passe = Non fitté (Le XX demande la rectification, le reste est standard)
		\item XX = Fitté sans l'arrêt (Puis tout est standard)
		\item Rectification = Fitté avec l'arrêt\\
		\end{itemize}
		
		Si le 2\trefle\ Stayman est contré :\\
		\begin{tabular}{cccc|l}
		1SA & - & 2\trefle & X &\\
		Passe &&&& Les autres mains (Typiquement des 4333)\\
		XX &&&& Pour jouer\\
		2\carreau &&&& Naturel\\
		2\coeur/\pique &&&& 4\pique/\coeur\ (Inversé pour mettre le X à l'entame)\\
		2SA &&&& Un arrêt solide\\
		3\trefle &&&& Les deux majeurs (Pas de Texas \& Le palier de 3 est chelemisant)\\
		\end{tabular}\\\\
		
		
		\SSection{Rubensohl}
		Sur un contre punitif, on joue Tout-Texas.
	
		Sur un contre artificiel ou 2\trefle\ naturel :
		\begin{itemize}
		\item On ignore l'intervention au palier de 2 (Le X de 2\trefle\ est Stayman)
		\item Le Texas impossible annonce un singleton dans la couleur sans le jeu nécessaire pour X
		\item Le palier de 3 est naturel fort\\
		\end{itemize}
		
		Sur une intervention 2\carreau/2\coeur/2\pique\ naturelle (De même sur 2\coeur/2\pique\ = 5\coeur/\pique\ + Une mineure 4$^{ème}$) :
		\begin{itemize}
		\item Les enchères jusqu'à 2\pique\ sont naturelles faibles (3SA est aussi naturel)
		\item Les enchères de 2SA à 3\coeur\ sont Texas (Texas Impossible = Stayman)
		\item Le contre, transformable, annonce du jeu
		\end{itemize}
		Sur 2\coeur/2\pique, 3\pique\ annonce l'autre majeure cinquième et l'arrêt (Par inférence, le Texas dénie l'arrêt)\\
		
		Sur une intervention de 3\trefle\ naturel, le X est d'appel et le reste Texas
		
		Sur une intervention de 3\carreau\ naturel, 3\coeur\ est Texas \pique\ et 3\pique\ est Texas \coeur\\
		
		Sur une intervention en Texas :
		\begin{itemize}
		\item Le contre est punitif
		\item La rectification est d'appel
		\item Le reste est Rubensohl\\
		\end{itemize}
		
		\SSection{Autres}
		\begin{tabular}{cccc|l}
		1SA & 2\trefle\ (Landy) & X && Punitif dans au moins une majeure OU 8H+ sans arrêt\\
		&& 2\coeur/2\pique\ && Singleton \pique/\coeur\ avec un 54 mineur\\
		&& 2SA/3\trefle && Texas \trefle/\carreau\\
		&& 3\carreau && Une majeure cinquième\\
		&& 3\coeur/3\pique && Chicane \pique/\coeur\ avec un 55 mineur\\
		\end{tabular}\\\\
		
		\begin{tabular}{cccc|l}
		1SA & 2\carreau\ (Unicolore Majeur) & X && Punitif dans au moins une majeure\\
		&& 2\coeur/2\pique\ && Naturel\\
		&& 2SA/3\trefle\ && Texas \trefle/\carreau\\
		\end{tabular}\\\\
		
		\begin{tabular}{cccc|l}
		1SA & 2\carreau\ (Unicolore Majeur) & - & 2\coeur/2\pique &\\
		- & Passe/2\pique & Enchère && Rubensohl\\
		\end{tabular}\\\\

		Sur 2SA bicolore mineur, on continue Stayman/Texas\\
	
\Chap{Ouverture de 2SA}
	\Section{Réponses}
	\begin{tabular}{cccc|l}
	2SA & - & 3\trefle && Stayman 4 paliers\\
	&& 3\carreau/\coeur && Texas \coeur/\pique\ (Rectification fittée sauf après passe)\\
	&& 3\pique/4\trefle && Texas \trefle/\carreau\\
	&& 3SA && Naturel\\
	&& 4\carreau && Bicolore majeur au moins 5/5\\
	&& 4\coeur/\pique && 5/4 mineur chelemisant ; Singleton \pique/\coeur\\
	&& 4SA/5SA && Quantitatif\\
	\end{tabular}\\
	NB : On ouvrira de 2SA avec 20-22 points (Avec 23+, on ouvrira de 2\trefle\ FM)\\

	\Section{Développements}
		\SSection{Stayman 4 paliers}
		\begin{tabular}{cccc|l}
		2SA & - & 3\trefle & - &\\
		3\carreau & - & 3\coeur/3\pique && Chassé croisé (Sur 3SA on remet 4\coeur/4\pique\ avec un 6/4 fort)\\
		&& 4\trefle/4\carreau && Naturel chelemisant\\
		\end{tabular}\\\\

		\begin{tabular}{cccc|l}
		2SA & - & 3\trefle & - &\\
		3SA & - & 4\carreau && Texas \coeur\ (Puis BW si chelemisant)\\
		&& 4\coeur && Texas \pique\ (Puis BW si chelemisant)
		\end{tabular}\\\\

		\SSection{Texas majeur}
		Lorsqu'il s'agit d'une intervention sur un barrage ou d'un réveil, la rectification est obligatoire\\

		\begin{tabular}{cccc|l}
		2SA & - & 3\carreau/3\coeur & - &\\
		3SA & - & 4\trefle/4\carreau && Naturel chelemisant OU Bicolore 5/5 de manche\\
		&& 4\coeur && Pour jouer si Texas \coeur ; Re-Texas si Texas \pique\\
		&& 4\pique && BW \coeur\ si Texas \coeur ; Pour jouer si Texas \pique\\
		\end{tabular}\\\\

\Chap{Ouverture de 2\trefle\ FM}
	\Section{Réponses}
	\begin{tabular}{cccc|l}
	2\trefle & - & 2\carreau && Pas d'As ; Maximum 1 Roi\\
	&& 2\coeur && 1 As sans Roi OU 4 Rois\\
	&& 2\pique && 1 As et 1 Roi\\
	&& 2SA && Pas d'As ; Exactement 2 Rois\\
	&& 3\trefle && 1 As et 2+ Roi OU 3 Rois sans le Roi \trefle\\
	&& 3\carreau/\coeur/\pique && 2 As CRM sans Roi OU 3 Rois sans le Roi \carreau/\coeur/\pique\\
	&& 3SA && 2 As et 1+ Rois\\
	&& 4\trefle/\carreau/\coeur/\pique && 3 As sans l'As nommé\\
	&& 4SA && 4 As\\
	\end{tabular}\\\\

		\SSection{Réponses après un X}
		\begin{itemize}
		\item Passe = Pas d'As
		\item 2\carreau\ = 1 As mineur sans Roi
		\item 2\coeur\ = 1 As majeur sans Roi
		\item Pas de changement pour le reste\\
		\end{itemize}
	
	\Section{Développement}
		\SSection{Control Asking Bid \& Trump Asking Bid}
		Toutes les enchères à SA qui suivent un fit sont des TAB. De même des rectifications de Texas.\\
			
		Le Trump Asking Bid est une demande de qualité d'atout dont les réponses sont par palier :\\
		RIEN ; 1 GH ; +1 carte de fit ; 1 GH +1 carte de fit ; 2GH ; 2 GH +1 carte de fit\\
	
		Le Control Asking Bid est une demande de contrôle en dehors de l'atout :\\
		RIEN ; 2$^{ème}$ tour ; 1$^{er}$ tour ; RDx ; ARx ; ARD ou AR sec (Ou As sec/Chicane et assez d'atout)\\
	
		Lorsqu'un As est nié ou que le CAB est répété, on parle de CAB de 2$^{ème}$ tour :\\
		RIEN ; Doubleton ; Dame ; 2$^{ème}$ tour ; RDx ; Chicane\\
		
		\SSection{Les fits}
		Les fits donnés par le répondant :
		\begin{itemize}
		\item Encourageant quand le fit est donné en dessous du palier de la manche
		\item Splinter possible dans 3 atouts (Pas de Splinter dans un Roi sec)\\
		L'ouvreur peut rechercher la présence d'une chicane dans la couleur en annonçant cette couleur
		\item Si le fit est non forcing, le saut à SA est un fit encourageant
		\item Après 3 couleurs naturelles, la quatrième couleur est un fit encourageant
		\end{itemize}
		Les fits donnés par l'ouvreur le sont soit directement soit par un CAB (à saut) ou un TAB\\

		\SSection{Séquence spéciale}
		\begin{tabular}{cccc|l}
		2\trefle & - & 2\carreau & - &\\
		3\coeur/3\pique &&&& Unicolore Autonome\\
		\end{tabular}\\\\
		
		\begin{tabular}{cccc|l}
		2\trefle & - & 2\carreau & - &\\
		3\coeur/3\pique & - & 3SA && Pas de Roi (Puis conclusion ou CAB de 2ème tour)\\
		&& 4\trefle/4\carreau/4\coeur/4\pique && Roi de \trefle/\carreau/\coeur/\pique\ (Puis conclusion ou CAB de 2ème tour)\\
		\end{tabular}\\

\Chap{Ouverture de 2\carreau/2\coeur/2\pique\ et 3SA/4\trefle/4\carreau}
	\Section{2\carreau\ Multi}
	Version faible : 2\carreau\ est un unicolore majeur faible indéterminé\\

	\begin{tabular}{cccc|l}
	2\carreau & - & 2\coeur/2\pique/3\coeur/3\pique && Passe ou Corrige\\
	&& 2SA && Relais\\
	&& 3\trefle && Naturel ; Pour jouer\\
	&& 3\carreau && Forcing avec l'autre majeure présumée\\
	&& 4\trefle && Nomme ta majeur en Texas\\
	&& 4\carreau && Nomme ta majeur\\
	&& 4\coeur/4\pique && Naturel\\
	\end{tabular}\\\\

	\begin{tabular}{cccc|l}
	2\carreau & - & 2SA & - &\\
	3\trefle/3\carreau &&&& Minimum \coeur/\pique\\
	3\coeur/3\pique &&&& Maximum \pique/\coeur\\
	\end{tabular}\\\\
	
	\begin{tabular}{cccc|l}
	2\carreau & - & 3\carreau & - &\\
	3\coeur &&&& Unicolore \coeur\ sans 3\pique\\
	3\pique &&&& Unicolore \pique\ sans 3\coeur\\
	4\trefle &&&& Unicolore \coeur\ avec 3\pique\\
	4\carreau &&&& Unicolore \pique\ avec 3\coeur\\
	\end{tabular}\\\\

	\begin{tabular}{cccc|l}
	2\carreau & X & Passe && Du carreau ; Accepte 2\carreau\ X\\
	&& XX && Pour jouer 2\coeur\ ou 2\pique\\
	&& 2\coeur && Pour jouer 2\coeur\ ou 3\pique\\
	&& 2\pique/3\coeur/3\pique && Passe ou Corrige\\
	&& 4\trefle && Nomme ta majeure en Texas\\
	&& 4\carreau && Nomme ta majeure\\
	&& 4\coeur/4\pique && Naturel\\
	\end{tabular}\\
	En cas d'intervention naturelle majeure, le contre est passe ou corrige\\	
	En cas d'intervention naturelle par 3\trefle, les enchères 4\trefle\ et 4\carreau\ gardent leur sens\\	
	En cas d'intervention naturelle par 3\carreau, 4\trefle\ est non forcing et 4\carreau\ demande la majeur\\

	\Section{2\coeur\ Non-Vulnérable}
	2\coeur\ est un bicolore majeur faible au moins 4/4\\

	\begin{tabular}{cccc|l}
	2\coeur & - & Passe/2\pique && Pour les jouer\\
	&& 2SA && Relais encourageant\\
	&& 3\trefle/3\carreau && 6 cartes solides ; Pour les jouer\\
	&& 3\coeur/3\pique && Barrage\\
	\end{tabular}\\\\

	\begin{tabular}{cccc|l}
	2\coeur & - & 2SA & - &\\
	3\trefle &&&& 4\coeur/4\pique\ (3\carreau\ demande la force)\\
	3\carreau &&&& 5\coeur/4\pique\ Minimum\\
	3\coeur &&&& 5\pique/4\coeur\ Minimum\\
	3\pique &&&& 5\coeur/4\pique\ Maximum\\
	3SA &&&& 5\pique/4\coeur\ Maximum\\
	4\trefle &&&& 5\coeur/5\pique\ Maximum ; Singleton \trefle\\
	4\carreau &&&& 5\coeur/5\pique\ Maximum ; Singleton \carreau\\
	4\coeur &&&& 5\coeur/5\pique\ Minimum\\
	\end{tabular}\\

	\Section{2\coeur\ Vulnérable et 2\pique}
	2\coeur/\pique\ est un bicolore faible au moins 5\coeur/\pique\ et 4\trefle/4\carreau\\

	\begin{tabular}{cccc|l}
	2\coeur/\pique && 2SA && Relais descriptif\\
	&& 3\trefle/4\trefle/5\trefle && Passe ou Corrige\\
	&& 3\carreau && 5 cartes dans l'autre majeur\\
	&& 3\coeur/\pique/4\coeur/\pique && Barrage\\
	&& 3\pique && Naturel forcing\\
	&& 3SA && Pour jouer\\
	\end{tabular}\\\\

	\begin{tabular}{cccc|l}
	2\coeur/\pique & - & 2SA & - &\\
	3\trefle/3\carreau &&&& 5\coeur/\pique\ et 4\trefle/4\carreau\ Minimum\\
	3\coeur/3\pique &&&& 5\coeur/\pique\ et 4\trefle/4\carreau\ Maximum\\
	3SA &&&& 5\coeur/\pique\ et 4\trefle\ et 4\carreau\\
	4\trefle/4\carreau &&&& 5\coeur/\pique\ et 5\trefle/5\carreau\\
	\end{tabular}\\\\
	
	Sur les réponses de 3\trefle/3\carreau/3\coeur/3\pique:
	\begin{itemize}
	\item 3\coeur/\pique\ est pour les jouer
	\item 4\trefle/\carreau\ est chelemisant dans la mineure
	\item Le reste est chelemisant dans la majeure\\
	\end{itemize}
	
	Après X ou sur une ouverture en $3^{ème}$, 2SA devient un relais pour la mineur et 3\trefle/\carreau\ sont naturels\\

	\Section{3SA Gambling}
	\begin{tabular}{cccc|l}
	3SA & - & 4\trefle && Passe/Corrige\\
	&& 4\carreau && Relais descriptif\\
	&& 4\coeur/4\pique/5\trefle/5\carreau/6\trefle/6\carreau && Pour jouer\\
	&& 4SA && Nomme ta mineure\\
	\end{tabular}\\\\
	
	\begin{tabular}{cccc|l}
	3SA & - & 4\carreau & - &\\
	4\coeur/4\pique &&&& Singleton \pique/\coeur\ (4SA demande la mineure, le reste est pour jouer)\\
	4SA &&&& 7222\\
	5\trefle/5\carreau &&&& Singleton \carreau/\trefle\\
	\end{tabular}\\\\
	
	\Section{4\trefle/4\carreau\ Namyats}
	Il s'agit de barrage de 4\coeur/4\pique\ avec au plus une perdante à l'atout en face de la chicane
	
	Cela ne se joue qu'en première et deuxième position\\
	
	La collante est un relais qui demande une description\\
	\begin{tabular}{cccc|l}
	4\trefle & - & 4\carreau & - &\\
	4\coeur &&&& Une perdante à l'atout\\
	4SA &&&& 7222 sans perdante à l'atout\\
	4\pique/5\trefle/5\carreau &&&& Singleton \pique/\trefle/\carreau\ sans perdante à l'atout\\
	\end{tabular}\\
	De même pour les \pique\\
	
\Chap{Ouverture majeure}
	\Section{Fit majeur (Sans passe d'entrée)}
	\begin{tabular}{cccc|l}
	1\coeur & - & 1SA && Peut cacher un fit faible\\
	&& 2\coeur && 6-10 ; Fitté\\
	&& 2\pique && Fitté limite avec 4\pique\\
	&& 2SA && Fitté FM ; Main intéressante\\
	&& 3\trefle && Fitté limite par 4\\
	&& 3\carreau && Fitté limite par 3\\
	&& 3\coeur && Barrage\\
	&& 3\pique/4\trefle/4\carreau && Splinter\\
	&& 3SA && Fitté FM ; Main banale\\
	&& 4\coeur && Barrage\\
	\end{tabular}\\\\

	Même principe sur 1\pique, sauf l'enchère de 3\coeur, qui est une main limite, sans le fit \pique, avec 6\coeur\\

		\SSection{En cas d'intervention}
		Sur le contre, on jouera 2SA Truscott, 3SA Super-Truscott\\
		De plus, on distingue deux types de fit au palier de 2 :

		\begin{tabular}{cccc|l}
		1\coeur & X & 2\carreau && Fit constructif\\
		&& 2\coeur && Fit barrage\\
		\end{tabular}
		
		De même pour les \pique\\

		En cas d'intervention à la couleur, on jouera 2SA Truscott, le Cue-bid fitté par 4
		
		Dans les deux cas, les changements de couleur à saut sont naturels faibles non fitté. Sauf 4\trefle\ et 4\carreau\\

		\SSection{Développement après 1\coeur/\pique\ - 3\trefle}
		\begin{tabular}{cccc|l}
		1\coeur/\pique & - & 3\trefle & - &\\
		3\carreau &&&& Relais demande de courte\\
		\end{tabular}\\\\
		
		\begin{tabular}{cccc|l}
		1\coeur/\pique & - & 3\trefle & - &\\
		3\carreau & - & 3\coeur/\pique && Pas de courte\\
		&& 3\pique/4\trefle/4\carreau/4\coeur && Couleur de la courte\\
		\end{tabular}\\\\

		\SSection{2SA Forcing de manche}
		Le principe sera le même sur l'ouverture d'1\pique\\
		
		\begin{tabular}{cccc|l}
		1\coeur & - & 2SA & - &\\
		3\trefle &&&& 12-14\\
		3\carreau &&&& 12-14 ; 6\coeur\ et 1 courte (Sur 3\coeur, on nomme la couleur de la courte)\\
		3\coeur &&&& 15-17 sans courte\\
		3\pique &&&& 15-17 avec une courte (Sur 3SA, on nomme la couleur de la courte)\\
		3SA &&&& 18-19\\
		4\trefle/4\carreau &&&& Bicolore concentré\\
		4\coeur &&&& Fort indéterminé\\
		\end{tabular}\\\\
		
		\begin{tabular}{cccc|l}
		1\coeur & - & 2SA & - &\\
		3\trefle & - & 3\carreau & - &\\
		3\coeur &&&& Minimum sans courte\\
		3\pique/4\trefle/4\carreau &&&& Couleur de la courte\\
		3SA &&&& Maximum sans courte\\
		4\coeur &&&& 6\coeur\\
		\end{tabular}\\
	
\newpage
	\Section{Fit majeur (Après passe d'entrée)}
		\SSection{2\trefle\ Drury et inférence}
		Cette enchère regroupe les fits limites\\
		\begin{tabular}{cccc|l}
		& (-) & - & - &\\
		1\coeur/\pique & - & 2\trefle & - &\\
		2\carreau &&&& Demande de description\\
		2\coeur/\pique &&&& Pour jouer\\
		4\coeur/\pique &&&& Pour jouer\\
		\end{tabular}\\
		Après une ouverture d'1\pique, 2\coeur\ promet un jeu sans l'ouverture dans un 5/4\\

		\begin{tabular}{cccc|l}
		& (-) & - & - &\\
		1\coeur/\pique & - & 2\trefle & - &\\
		2\carreau & - & 2\coeur/\pique && Main limite fittée par 3 cartes\\
		&& 2SA && 4333 fitté par 4 cartes\\
		&& 3\coeur/\pique && Main limite fittée par 4 cartes\\
		\end{tabular}\\\\
		
		L'enchère de 2SA annonce alors une main limite fitté par 4 cartes avec un singleton
		
		La demande de singleton se fait à 3\trefle\ (Le singleton \trefle\ est à 3\coeur/\pique)\\

	\Section{Séquences particulières}
		\SSection{Cas des coeurs}
	
		Par inférence de 1\pique\ - 3\coeur, la séquence 		
		\begin{tabular}{cccc|l}
		1\pique & - & 2\coeur & - &\\
		2\pique & - & 3\coeur && \\
		\end{tabular}
		est forcing de manche\\
	
		\SSection{Rebid à 2SA de l'ouvreur FM}
		\begin{tabular}{cccc|l}
		1\coeur & - & 1\pique & - &\\
		2SA & - & 3\trefle && Relais descriptif\\
		&& 3\carreau && 5\pique\ forcing manche\\
		&& 3\coeur && Chelemisant\\
		&& 3\pique && 6\pique\ forcing manche\\
		\end{tabular}\\
		Même principe dans les autres séquences.\\
		
		\begin{tabular}{cccc|l}
		1\coeur & - & 1\pique/1SA & - &\\
		2SA & - & 3\trefle & - &\\
		3\carreau &&&& Bicolore à saut ; 5\coeur/4\carreau\\
		3\coeur &&&& Bicolore à saut ; 5\coeur/4\trefle\\
		3SA &&&& (Si 1\pique) 5332 avec 3\pique ; (Si 1SA) 5332 avec un petit doubleton \pique\\
		\end{tabular}\\\\
		
		\begin{tabular}{cccc|l}
		1\pique & - & 1SA & - &\\
		2SA & - & 3\trefle & - &\\
		3\carreau &&&& Bicolore à saut ; 5\pique/4\carreau\\
		3\coeur &&&& Bicolore à saut ; 5\pique/4\coeur\\
		3\pique &&&& Bicolore à saut ; 5\pique/4\trefle\\
		3SA &&&& 5332 avec 3\coeur\ (Puis Texas \coeur\ si 5\coeur)\\
		\end{tabular}\\\\

		Par inférence, les bicolores à saut 1\coeur\ - 1\pique/1SA - 3\trefle/3\carreau\ et 1\pique\ - 1SA - 3\carreau/3\coeur\ sont 5/5
		
\newpage
		\SSection{El Gazzili}
		\underline{Principe}
		
		\begin{tabular}{cccc|l}
		1? & - & 1? & - &\\
		2\trefle & - & 2\carreau && Forcing ; 8+ points\\
		&& Autre && 7-\\
		\end{tabular}\\\\
		
		\underline{Après 1\coeur\ - 1\pique}

		\begin{tabular}{cccc|l}
		1\coeur & - & 1\pique & - &\\
		2\trefle & - & 2\carreau & - &\\
		2\coeur &&&& 12-14 ; Au moins 5\coeur/4\trefle\\
		2\pique &&&& Les mains de 3\trefle/3\carreau/3\coeur/3SA avec 3\pique\ (Relais à 2SA)\\
		2SA &&&& 15-16 ; 5332 (Développement en Texas)\\
		3\trefle &&&& 15-17 ; 5\coeur/4\trefle\ sans 3\pique\\
		3\carreau &&&& F.I. \coeur\ sans 3\pique\\
		3\coeur &&&& 15-17 ; 6\coeur/4\trefle\ sans 3\pique\\
		3SA &&&& 17-18 ; 5332 sans 3\pique\\
		\end{tabular}\\\\

		\begin{tabular}{cccc|l}
		1\coeur & - & 1\pique & - &\\
		2\trefle & - & 2\carreau & - &\\
		2\coeur & - & 2\pique/3\trefle/3SA && Pour jouer\\
		&& 2SA && Propositionnel\\
		&& 3\carreau && 4$^{ème}$ Forcing\\
		&& 3\coeur && Chelemisant \coeur\\
		&& 3\pique && 6+\pique\ ; Forcing manche (Départ en contrôle si fitté, 3SA sinon)\\
		\end{tabular}\\\\

		\begin{tabular}{cccc|l}
		1\coeur & - & 1\pique & - &\\
		2\trefle & - & 2\carreau & - &\\
		3\trefle & - & 3\carreau && 4$^{ème}$ Forcing\\
		&& 3\coeur && Chelemisant \coeur\\
		&& 3\pique && 6+\pique\ ; Forcing manche (Départ en contrôle si fitté, 3SA sinon)\\
		\end{tabular}\\\\

		\underline{Après 1\coeur\ - 1SA}

		\begin{tabular}{cccc|l}
		1\coeur & - & 1SA & - &\\
		2\trefle & - & 2\carreau & - &\\
		2\coeur &&&& 12-14 ; Au moins 5\coeur/4\trefle\\
		2\pique &&&& 15-17 ; Au moins 5\coeur/4\pique\\
		2SA &&&& 15-16 ; 5332 (Développement en Texas *)\\
		3\trefle &&&& 15-17 ; 5\coeur/4\trefle\\
		3\carreau &&&& F.I. \coeur\\
		3\coeur &&&& 15-17 ; 6\coeur/4\trefle\\
		3SA &&&& 17-18 ; 5332\\
		\end{tabular}\\
		(*) Le Texas dans l'ouverture est un Bicolore Mineur ; Le Texas \pique\ est une demande d'arrêt\\

		\underline{Après 1\pique\ - 1SA}

		\begin{tabular}{cccc|l}
		1\pique & - & 1SA & - &\\
		2\trefle & - & 2\carreau & - &\\
		2\coeur &&&& Les mains de 3\trefle/3\carreau/3\pique/3SA avec 3\coeur\ (Relais à 2SA)\\
		2\pique &&&& 12-14 ; Au moins 5\pique/4\trefle\\
		2SA &&&& 15-16 ; 5332 (Développement en Texas *)\\
		3\trefle &&&& 15-17 ; Au moins 5\pique/4\trefle\ sans 3\coeur\\
		3\carreau &&&& F.I. \pique\ sans 3\coeur\\
		3\pique &&&& 15-17 ; Au moins 6\pique/4\trefle\ sans 3\coeur\\
		3SA &&&& 17-18 ; 5332 sans 3\coeur\\
		\end{tabular}\\
		(*) Le Texas dans l'ouverture est un Bicolore Mineur

\Chap{Ouverture mineure}
	\Section{Soutien mineur inversé}
		\SSection{Principe}
		\begin{tabular}{cccc|l}
		1\trefle & - & 2\trefle && Soutien FM (Possiblement 4\coeur/4\pique)\\	
		&& 2\carreau && Soutien limite (9-11HL)\\
		&& 3\trefle && Soutien barrage (6-8 HL)\\
		\end{tabular}\\
		Après passe, le SMI n'est plus FM et dénie 4\coeur/4\pique\ (On nomme ses arrêts naturellement)\\
	
		\begin{tabular}{cccc|l}
		1\trefle & - & 2\trefle & - &\\
		2\carreau/2\coeur/2\pique &&&& Naturel irrégulier (dès 12 points)\\
		2SA &&&& 12-14 OU 18-19 ; Régulier\\
		3\trefle &&&& Unicolore 6$^{ème}$ (Ou 5$^{ème}$ avec un petit doubleton majeur)\\
		3\carreau/3\coeur/3\pique &&&& Bel unicolore 6$^{ème}$ ; Splinter\\
		3SA &&&& 15-17 ; 4441 Singleton \carreau\\
		4\trefle &&&& Chelemisant \trefle\\
		\end{tabular}\\\\

		\begin{tabular}{cccc|l}
		1\trefle & - & 2\carreau & - &\\	
		2\coeur &&&& Arrêt \coeur\ et \carreau\\
		2\pique &&&& Arrêt \pique\ et \carreau\\
		2SA &&&& Arrêt \coeur\ et \pique\\
		3\trefle &&&& Pour jouer\\
		3\carreau/3\coeur/3\pique &&&& Splinter\\
		3SA &&&& Pour jouer\\
		4\trefle &&&& Chelemisant \trefle\\
		\end{tabular}\\\\

		\begin{tabular}{cccc|l}
		1\carreau & - & 2\carreau && Soutien FM (Possiblement 4\coeur/4\pique)\\
		&& 3\trefle && Soutien limite (9-11HL)\\
		&& 3\carreau && Soutien barrage (6-8 HL)\\
		\end{tabular}\\\\

		\begin{tabular}{cccc|l}
		1\carreau & - & 2\carreau & - &\\
		2\coeur/2\pique/3\trefle &&&& Naturel irrégulier (dès 12 points)\\
		2SA &&&& 12-14 OU 18-19 ; Régulier\\
		3\carreau &&&& Unicolore 6$^{ème}$ (Ou 5$^{ème}$ avec un petit doubleton majeur)\\
		3\coeur/3\pique &&&& Bel unicolore 6$^{ème}$ ; Splinter\\
		3SA &&&& 15-17 ; 4441 Singleton \trefle\\
		4\trefle &&&& Bel unicolore 6$^{ème}$ ; Splinter (Chelemisant)\\
		4\carreau &&&& Chelemisant \carreau\\
		\end{tabular}\\\\
		
		\SSection{Problème des carreaux}
		\begin{tabular}{cccc|l}
		1\carreau & - & 3\trefle & - &\\	
		3\carreau &&&& Pour jouer\\
		3\coeur &&&& Arrêt \coeur\ et \carreau\ (3\pique\ demande d'arrêt)\\
		3\pique &&&& Arrêt \pique\ et \carreau\\
		3SA &&&& Arrêt \coeur\ et \pique\\
		4\trefle/4\coeur/4\pique &&&& Splinter\\
		4\carreau &&&& Chelemisant \carreau\\
		\end{tabular}\\\\
	
		\begin{tabular}{cccc|l}
		1\carreau & - & 3\carreau & - &\\
		3\coeur &&&& Arrêt \coeur\ et \trefle\ (3\pique\ demande d'arrêt)\\
		3\pique &&&& Arrêt \pique\ et \trefle\\
		3SA &&&& Arrêt \coeur\ et \pique\\
		\end{tabular}\\

		\SSection{Développement sur 2SA}
		\begin{tabular}{cccc|l}
		1\trefle/\carreau & - & 2\trefle/\carreau & - &\\
		2SA & - & 3\trefle && Relais pour 3\carreau\\
		&& 3\carreau/3\coeur && 4\coeur/4\pique\\
		&& 3\pique && Bicolore 5\trefle/5\carreau\\
		&& 3SA && Pour jouer\\
		&& 4\trefle/\carreau && Chelemisant\\
		\end{tabular}\\\\

		\begin{tabular}{cccc|l}
		1\trefle/\carreau & - & 2\trefle/\carreau & - &\\
		2SA & - & 3\trefle & - &\\
		3\carreau & - & 3\coeur/\pique && Singleton \pique/\coeur\\	
		&& 3SA && Singleton \carreau/\trefle\ (Autre mineur)\\
		&& 4\trefle/\carreau && Singleton \carreau/\trefle\ (Autre mineur) ; Chelemisant\\
		\end{tabular}\\\\

		\begin{tabular}{cccc|l}
		1\trefle/\carreau & - & 2\trefle/\carreau & - &\\
		2SA & - & 3\carreau & - &\\
		3\coeur &&&& 18-19 ; Fitté OU Problème d'arrêt \pique\\
		3SA &&&& 12-14 ; Non fitté\\
		4\coeur &&&& 12-14 ; Fitté\\
		4SA &&&& 18-19 ; Non fitté\\
		\end{tabular}\\
		De même pour les \pique\\

		\SSection{Développement sur 3SA}
		\begin{tabular}{cccc|l}
		1\trefle/\carreau & - & 2\trefle/\carreau & - &\\
		3SA & - & 4\trefle && Chelemisant \trefle/\carreau\\
		&& 4\carreau/4\coeur && 4\coeur/4\pique\ (Le fit est certain)\\
		&& 4SA && Quantitatif\\
		\end{tabular}\\\\

		\SSection{Réactions en cas d'interventions}
		Après un contre ou une intervention, on ne joue plus le soutien mineur inversé\\
		Après un contre, le XX comprend les mains fittées en mineur FM\\
		Sur une intervention, le Cue-Bid comprend les mains fittées en mineur FM\\
		
		\begin{tabular}{cccc|l}
		1\trefle/\carreau & 1\pique & X && Spoutnik ; Dénie 5\coeur\\
		&& 2\trefle/\carreau && 8-10 ; 5\coeur\ (Rodrigue)\\
		&& 3\trefle/\carreau && Barrage\\
		\end{tabular}\\\\
		
		\begin{tabular}{cccc|l}
		1\trefle/\carreau & X/1\carreau/1\coeur & 2\trefle/\carreau && Naturel compétitif\\
		&& 3\trefle/\carreau && Barrage\\
		\end{tabular}\\\\

		\begin{tabular}{cccc|l}
		1\carreau & 2\trefle & X && 8-10 ; 5\pique\ OU 4\coeur/4\pique\ OU 4\coeur/\pique\ et du \carreau\ OU 11+\\
		&& 2\carreau && 8-10 ; 5\coeur\\
		&& 3\carreau && Barrage\\
		\end{tabular}\\

\newpage
		\SSection{Séquence 1\carreau\ - 2\trefle}
		\begin{tabular}{cccc|l}
		1\carreau & - & 2\trefle & - &\\
		2\carreau &&&& Unicolore 6$^{ème}$ OU Bicolore de première zone\\
		2\coeur/2\pique &&&& Naturel irrégulier (dès 15 points) \\
		2SA &&&& 12-14 OU 18-19 ; Régulier\\
		3\trefle &&&& Naturel irrégulier (dès 15 points)\\
		3SA &&&& 15-17 ; 4441 Singleton \trefle\ (Puis tout Texas)\\
		\end{tabular}\\\\

		\begin{tabular}{cccc|l}
		1\carreau & - & 2\trefle & - &\\
		2\carreau & - & 2\coeur/2\pique && 4\coeur/\pique\ OU Un arrêt pour 3SA\\
		&& 2SA/3\trefle && Naturel limite\\
		&& 3\carreau && Chelemisant \carreau\\
		&& 3\coeur/3\pique && Fit \carreau\ ; Splinter \coeur/\pique\\
		&& 3SA && Naturel\\
		\end{tabular}\\\\

		\begin{tabular}{cccc|l}
		1\carreau & - & 2\trefle & - &\\
		2\carreau & - & 2\coeur/2\pique & - &\\
		2SA &&&& Arrêt \pique/\coeur\ sans fit\\
		3\coeur/3\pique &&&& Fit \coeur/\pique\ sans l'arrêt \pique/\coeur\\
		3SA &&&& Fit \coeur/\pique\ avec l'arrêt \pique/\coeur\\
		\end{tabular}\\\\

		\begin{tabular}{cccc|l}
		1\carreau & - & 2\trefle & - &\\
		2SA & - & 3\trefle && Relais pour 3\carreau\\
		&& 3\carreau/3\coeur && 4\coeur/4\pique\\
		&& 3\pique && Proposition de chelem à \trefle\ (3SA décourageant)\\
		&& 3SA && Pour jouer\\
		&& 4\trefle/4\carreau && Naturel chelemisant\\
		\end{tabular}\\\\

		\begin{tabular}{cccc|l}
		1\carreau & - & 2\trefle & - &\\
		2SA & - & 3\trefle & - &\\
		3\carreau & - & 3\coeur && Singleton \pique\\
		&& 3\pique && Singleton \coeur\\	
		&& 3SA && Singleton \carreau\\
		\end{tabular}\\\\

		\begin{tabular}{cccc|l}
		1\carreau & - & 2\trefle & - &\\
		2SA & - & 3\carreau & - &\\
		3\coeur &&&& 18-19 ; Fitté OU Problème d'arrêt \pique\\
		3SA &&&& 12-14 ; Non fitté\\
		4\coeur &&&& 12-14 ; Fitté\\
		4SA &&&& 18-19 ; Non fitté\\
		\end{tabular}\\
		De même pour les \pique\\

\newpage
	\Section{Autres conventions}
		\SSection{Rebid à 2SA de l'ouvreur}
		\begin{tabular}{cccc|l}
		1\trefle/\carreau & - & 1\coeur & - &\\
		2SA & - & 3\trefle && Texas \carreau\ (3SA négatif)\\
		&& 3\carreau && Texas \coeur\\
		&& 3\pique && Texas \trefle\ (3SA négatif)\\
		\end{tabular}\\\\
	
		\begin{tabular}{cccc|l}
		1\trefle/\carreau & - & 1\pique & - &\\
		2SA & - & 3\trefle && Texas \carreau\ (3SA négatif)\\
		&& 3\carreau && Texas \coeur\ donc 5\pique/4\coeur\\
		&& 3\coeur && Texas \pique\\
		&& 3\pique && Texas \trefle\ (3SA négatif)\\
		\end{tabular}\\\\
		
		Pour les séquences 1 mineur - 1\coeur\ - 2SA - 3\carreau\ - 3SA et 1 mineur - 1\pique\ - 2SA - 3\coeur\ - 3SA, on développera comme sur les séquences respectives 2SA - 3\carreau\ - 3SA et 2SA - 3\coeur\ - 3SA en standard\\
	
		\begin{tabular}{cccc|l}
		1\trefle/\carreau & - & 1\pique & - &\\
		2SA & - & 3\carreau & - &\\
		3SA & - & 4\carreau && Bicolore au moins 5\pique/5\coeur\ (l'ouvreur dira 4\pique\ sans fit \coeur)\\
		&& 4\coeur && Texas \pique\\
		&& 4\pique && Pour jouer\\
		&& 4SA && Quantitatif\\
		\end{tabular}\\\\

		\SSection{Convention Bessis}
		\begin{tabular}{cccc|l}
		1\trefle/\carreau & - & 2\coeur && 6-10 ; 5\pique/4\coeur\ (Possiblement 5/5)\\
		\end{tabular}\\
		ATTENTION : Par inférence, 1\trefle/\carreau\ - 1\pique\ - 1SA - 2\coeur\ promet un 5/4 dans une main limite\\
		De plus, 1\trefle/\carreau\ - 1\pique\ - 1SA - 3\coeur\ promet un 5/5 dans une main limite\\
	
		\begin{tabular}{cccc|l}
		1\trefle/\carreau & - & 2\coeur & - &\\
		2SA & - & 3\trefle && 5/4 minimum\\
		&& 3\carreau && 5/5 minimum\\
		&& 3\coeur && 5/4 maximum\\
		&& 3\pique && 5/5 maximum\\
		\end{tabular}\\\\
	
		La séquence 1\trefle/\carreau\ - 2\pique\ promet par contre 6\pique\ dans une main faible\\
		
		Après intervention, il n'y a pas de changements sur le sens de ces enchères\\
		
		\SSection{Défense contre les Michaëls}
		Le principe général est le suivant :
		\begin{itemize}
		\item 2SA : Fitté limite
		\item Cue bid de la moins cher : La couleure non nommée, FM
		\item Cue bid de la plus chère : Fittée, FM
		\item Autre : Naturel
		\end{itemize}
	
		NB : Cette défense est également valable en cas d'ouverture majeure\\
		De plus, lorsque l'enchère de 3\trefle\ n'est pas disponible, l'enchère de X la remplace\\
	
\Chap{Intervention}
	\Section{Sur 1SA Fort}
	\underline{En 2$^{ème}$ ou 4$^{ème}$ position}
	
	\begin{tabular}{cccc|l}
	1SA & X &&& Mineur cinquième et 4/3 majeur\\
	& 2\trefle &&& Landy 5/4 majeur (2\carreau\ demande de la plus longue)\\
	& 2\carreau &&& Multi\\
	& 2\coeur/2\pique &&& 5\coeur/\pique\ et une mineure 4$^{ème}$ (2SA demande de la mineure)\\
	& 2SA &&& Bicolore 5\trefle/5\carreau\\
	& 3\trefle/3\carreau &&& Naturel\\
	\end{tabular}\\
	Sur le X, 2\trefle\ demande la mineure et 2\carreau\ demande la majeure\\

	\underline{Après un Texas}
	
	\begin{itemize}
	\item X = D'entame (Le contre est d'appel en réveil après la rectification)
	\item Rectification = Bicolore 5/4 majeur/mineur
	\item 2SA = Bicolore 5/5 mineur (De même en réveil après la rectification)
	\item Autre = Unicolore naturel (De même en réveil après la rectification)\\
	\end{itemize}

	\Section{Sur 1SA Faible}
	\begin{tabular}{cccc|l}
	1SA & X &&& Jeu régulier ou Unicolore puissant\\
	& 2\trefle &&& Landy\\
	& 2\carreau/2\coeur/2\pique/3\trefle &&& Texas\\
	& 2SA &&& Bicolore mineur\\
	\end{tabular}\\
	Le X promet 1 point de plus que le maximum de la zone du SA faible (15 H vs 12-14 ; 13 H vs 10-12)\\
	Sur le Texas, la rectification indique un jeu faible et ne promet pas le fit\\

	\begin{tabular}{cccc|l}
	1SA & X & - & Passe & Régulier sans majeure 5$^{ème}$\\
	&&& 2\trefle/2\carreau/2\coeur/2\pique & 0-8 ; Naturel\\
	&&& 2SA/3\trefle/3\carreau/3\coeur & 9+ ; Texas\\
	\end{tabular}\\\\

	\begin{tabular}{cccc|l}
	1SA & 2\trefle & - & 2\carreau & Relais\\
	&&& 2\coeur/2\pique & Préférence\\
	&&& 2SA & Cue-bid\\
	&&& 3\coeur/3\pique & Proposition\\
	&&& 4\coeur/4\pique & Pour les jouer\\
	\end{tabular}\\\\

	En cas d'enchère du n°3 :
	\begin{itemize}
	\item Si c'est une enchère naturelle, X est punitif et 2SA est Cue-bid
	\item Si c'est une enchère en Texas, X montre du jeu dans la couleur
	\end{itemize}
	Il n'y a pas de contre d'entame (Les contres annonceront du jeu)\\	

	\Section{Interventions \& Bicolores}
		Vulnérable, nos barrages au palier de $2$ sont $10-14$\\
	
		\SSection{Bicolore en 2$^{ème}$ position}
		On joue les Michaëls précisés avec deux subtilités :
		\begin{itemize}
		\item 1\trefle\ 2\trefle\ est un 5/4 majeur de 8-14 points
		\item 1\carreau\ 3\trefle\ est non forcing\\
		\end{itemize}

		\SSection{En 4$^{ème}$ position}
		Sur 1x - -, on ne joue pas les Michaëls précisés\\
		Le Cue-bid d'1\coeur/\pique\ annonce l'autre majeure et une mineure indéterminée

\newpage
	\Section{Intervention sur un barrage}
		\SSection{Sur 2\coeur/\pique\ faible}
		\begin{tabular}{cccc|l}
		2\coeur & X &&& Appel\\
		& 2\pique/3\trefle/3\carreau &&& Naturel\\
		& 2SA &&& 16-18 ; Naturel\\
		& 3SA &&& Pour jouer\\
		& 3\coeur &&& 5\trefle/5\carreau\\
		& 4\trefle &&& 5\trefle/5\pique\\
		& 4\carreau &&& 5\carreau/5\pique\\
		\end{tabular}\\\\
	
		\begin{tabular}{cccc|l}
		2\pique & X &&& Appel\\
		& 2SA &&& 16-18 ; Naturel\\
		& 3\trefle/3\carreau/3\coeur &&& Naturel\\
		& 3\pique &&& 5\trefle/5\carreau\\
		& 4\trefle &&& 5\trefle/5\coeur\\
		& 4\carreau &&& 5\carreau/5\coeur\\
		\end{tabular}\\\\

		\begin{tabular}{cccc|l}
		2\coeur/\pique & X & - & 2SA & 8-10 ; Toute main OU 11+ ; Avec l'autre majeure 4$^{ème}$\\
		&&& 2\pique/3\trefle/3\carreau/3\coeur & 0-7 ; Naturel\\
		&&& 3\coeur/\pique & 11+ ; Sans l'autre majeure 4$^{ème}$\\
		&&& 3\pique & 8-10 ; 5\pique\\
		&&& 3SA & Naturel avec l'arrêt\\
		\end{tabular}\\\\

		\begin{tabular}{cccc|l}
		2\coeur/\pique & X & - & 2SA &\\
		- & 3\trefle/3\carreau &&& 12-15 ; Meilleure mineure\\
		& 3\coeur &&& 12-15 ; 5\coeur\ moches (Sur 2\pique)\\
		& 3\coeur/3\pique &&& 16+ ; Demande d'arrêt OU Main forte\\
		& 3\pique &&& Main forte à pique (Sur 2\coeur)\\
		& 3SA &&& 16+ ; Avec l'arrêt\\
		\end{tabular}\\\\

		\SSection{Sur 2\carreau\ multi}
		\begin{tabular}{cccc|l}
		2\carreau & X &&& Contre d'appel court à \pique\\
		& 2\coeur &&& Contre d'appel court à \coeur\\
		& 2\pique &&& Naturel\\
		& 2SA &&& Naturel\\
		& 3\trefle/3\carreau &&& Naturel\\
		& 3\coeur/3\pique &&& Naturel 6+\coeur/\pique\\
		& 3SA &&& Tendance Gambling\\
		\end{tabular}\\\\
		
		\begin{tabular}{cccc|l}
		2\carreau & -  & 2\coeur & X & Contre d'appel court à \coeur\\
		&&& 2\pique & Naturel\\
		&&& 2SA & Naturel\\
		&&& 3\trefle/3\carreau & Naturel\\
		&&& 3\coeur/3\pique & Naturel 6+\coeur/\pique\\
		&&& 3SA & Tendance Gambling\\
		\end{tabular}\\
		Même principe sur les autres réponses\\
		
		\begin{tabular}{cccc|l}
		2\carreau & -  & 2\coeur & - &\\
		- & X &&& 13-15 ; Main régulière\\
		& 2\pique &&& Réveil Naturel 5+\pique\\
		& 2SA &&& Appel aux mineures\\
		& 3\trefle/3\carreau &&& Réveil naturel\\
		\end{tabular}\\\\
		De même sur 2\carreau\ - 2\pique\ - - et sur 2\carreau\ - 2\coeur\ - 2\pique\\
		
\Chap{Conventions particulières}
	\Section{Blackwood 5 clés}
		\SSection{Généralités}
		Sur un Blackwood, les réponses se font en 41/30/2\\
		L'annonce d'une chicane ne doit pas dépasser le petit chelem et dénie la Dame d'atout\\
		Sur un Blackwood d'exclusion, les réponses se font en 30/41/2\\

		\underline{Demande de la Dame d'atout}

		La collante de 5\trefle/5\carreau\ est une demande de la Dame d'atout dont les réponses sont
		\begin{itemize}
		\item 1$^{er}$ palier : Pas la Dame
		\item 2$^{ème}$ palier : Dame d'atout + 0/3 Roi(s)
		\item 3$^{ème}$ palier : Dame d'atout + Roi le moins cher (ou les 2 Rois les plus chers)
		\item 4$^{ème}$ palier : Dame d'atout + Roi du milieu (ou les 2 Rois des extrêmes)
		\item 5$^{ème}$ palier : Dame d'atout + Roi le plus cher (ou les 2 Rois les moins chers)\\
		\end{itemize}		

		\underline{Demande des Rois}
		
		La collante de 5\coeur/5\pique\ et la surcollante de 5\trefle/5\carreau\ est une demande aux Rois dont les réponses sont 
		\begin{itemize}
		\item 1$^{er}$ palier : 0/3 Roi(s)
		\item 2$^{ème}$ palier : Roi le moins cher (ou les 2 Rois les plus chers)
		\item 3$^{ème}$ palier : Roi du milieu (ou les 2 Rois des extrêmes)
		\item 4$^{ème}$ palier : Roi le plus cher (ou les 2 Rois les moins chers)\\
		\end{itemize}

		\SSection{Réaction en cas d'intervention}
		En cas d'intervention sur le BW, on joue C0P1 :
		\begin{itemize}
		\item Contre = 0 clé (ou 3 clés)
		\item Passe = 1 clé (ou 4 clés)
		\item La collante = 2 clés\\
		\end{itemize}

		\SSection{Cas particuliers}
		Lorsque certaines réponses de la question à la dame d'atout ou aux rois dépassent le petit chelem, celles-ci sont regroupées dans la réponse de premier palier et une collante est possible pour les distinguer\\
		
		\underline{Exemple : Atout \coeur}

		\begin{tabular}{cccc|l}
		... & - & 4SA& - &\\
		5\carreau & - & 5\pique & - & Demande de la dame d'atout\\
		5SA &&&& Réponse du 1$^{er}$ palier ou du 5$^{ème}$ palier\\
		6\trefle &&&& Réponse du 2$^{ème}$ palier\\
		6\carreau &&&& Réponse du 3$^{ème}$ palier\\
		6\coeur &&&& Réponse du 4$^{ème}$ palier \\
		\end{tabular}\\
		Sur 5SA, on relais à 6\trefle\ pour distinguer le 1$^{er}$ palier (6\carreau) et le 5$^{ème}$ palier (6\coeur)\\

		\begin{tabular}{cccc|l}
		... & - & 4SA& - &\\
		5\carreau/5\pique & - & 5SA & - & Demande de roi\\
		6\trefle &&&& Réponse du 1$^{er}$ palier ou du 4$^{ème}$ palier\\
		6\carreau &&&& Réponse du 2$^{ème}$ palier\\
		6\coeur &&&& Réponse du 3$^{ème}$ palier\\
		\end{tabular}\\
		Sur 6\trefle, on relais à 6\carreau\ pour distinguer le 1$^{er}$ palier (6\coeur) et le 4$^{ème}$ palier (6\pique)\\

\newpage
	\Section{Double 2}
	\begin{tabular}{cccc|l}
	1\trefle/\carreau & - & 1\pique & - &\\
	1SA & - & 2\trefle & - & Relais\\
	2\carreau & - & Passe && Pour jouer\\
	&& 2\coeur && Main limite avec du \trefle\\
	&& 2\pique && Main limite distribuée avec 5\pique\ ou 6\pique\\
	&& 2SA && Main limite régulière avec 5\pique\\
	&& 3\trefle && Faible avec du \trefle\\
	&& 3\carreau && Main limite avec du \carreau\\
	&& 3\coeur && 6\pique/4\coeur\ Main limite\\
	&& 3\pique && 6\pique\ Main limite\\
	&& 3SA && Proposition entre 3SA et 4\pique\\
	\end{tabular}\\
	Même principe sur 1\trefle/\carreau\ - 1\coeur\\

	\begin{tabular}{cccc|l}
	1\trefle/\carreau & - & 1\pique & - &\\
	1SA & - & 2\carreau & - & Relais FM (Soit 5\pique ; Soit un problème d'arrêt)\\
	2\coeur &&&& 4\coeur\ (Ne dénie pas 3\pique)\\
	2\pique &&&& 3\pique\\
	2SA &&&& Arrêt \coeur\ et \carreau/\trefle\\
	3\trefle/\carreau &&&& Belle couleure, mauvais arrêts\\
	3\carreau/\trefle &&&& Arrêt pour SA\\
	\end{tabular}\\
	Même principe sur 1\trefle/\carreau\ - 1\coeur\ et 1\coeur\ - 1\pique\\

	\begin{tabular}{cccc|l}
	1\coeur & - & 1\pique & - &\\
	1SA & - & 2\trefle & - & Relais\\
	2\carreau & - & Passe && Pour jouer\\
	&& 2\coeur && Main limite avec du \trefle\\
	&& 2\pique && Main limite distribuée avec 5\pique\ ou 6\pique\\
	&& 2SA && Main limite régulière avec 5\pique\\
	&& 3\trefle && Faible avec du \trefle\\
	&& 3\carreau && Main limite avec du \carreau\\
	&& 3\pique && 6\pique\ Main limite\\
	&& 3SA && Proposition entre 3SA et 4\pique\\
	\end{tabular}\\\\
	
	\Section{Troisième forcing}
	\begin{tabular}{cccc|l}
	1\trefle & - & 1\coeur & - &\\
	2\trefle & - & 2\carreau & - & Troisième forcing\\
	2\coeur &&&& 3\coeur\\
	2\pique/3\carreau &&&& 14+ ; Un arrêt pour 3SA\\
	2SA &&&& Minimum ; Tous les arrêts\\
	3\trefle &&&& Minimum\\
	3SA &&&& 14+ ; Tous les arrêts\\
	\end{tabular}\\
	De même pour la séquence 1\carreau\ - 1\pique\ - 2\carreau\ - 2\coeur\\
		
	\begin{tabular}{cccc|l}
	1\trefle & - & 1\pique & - &\\
	2\trefle & - & 2\carreau && Troisième forcing sans 4\coeur\\
	&& 2\coeur && Troisième forcing avec 4\coeur\\
	\end{tabular}\\\\
		
	\begin{tabular}{cccc|l}
	1\trefle & - & 1\coeur & - &\\
	2\trefle & - & 2\carreau & - & Troisième forcing\\
	2\coeur & - & 2SA && Relais propositionnel\\
	&& 3\coeur &&Chelemisant\\
	\end{tabular}\\
	De même sur les autres séquences
	
\Chap{Contres}
	\Section{Contres compétitifs}
		\SSection{Barrage}
		4\coeur\ - X : Appel
		
		4\pique\ - X : Optionnel
		
		5\trefle\ - X : Optionnel (dégagement très rare)
		
		Passe - Passe - 5\trefle\ - X : Punitif
		
		3\trefle\ - Passe - 5\trefle\ - X : Optionnel
		
		Passe - Passe - 3\trefle\ - X - 5\trefle\ - Passe : Forcing\\
		
		1\pique\ - X - 3\pique\ - X : Appel
		
		1\pique\ - X - 3\pique\ - X - Passe - 4\trefle/\carreau\ - Passe - 4\coeur\ : 4\coeur\ et une mineure
		
		2\pique\ - X - 3\pique\ - X : Appel\\
		
		\SSection{Contre de l'ouvreur}
		1\carreau\ - Passe - 1\coeur\ - 2\trefle\ - X : 3\coeur\ et du jeu (Passe ne dénie pas les 3 cartes)
		
		1\pique\ - Passe - 1SA - 2\trefle\ - X : Appel
		
		1\pique\ - Passe - 2\trefle\ - 2\coeur\ - X : Appel avec une main intéressante
		
		1\carreau\ - Passe - 1SA - 2\pique\ - X : Appel
		
		1\coeur\ - Passe - 4\coeur\ - 4\pique\ - X : Punitif (forcing Passe)\\
		
		\SSection{Après action du partenaire => Punitif si deux couleurs nommées}
		1\pique\ - Passe - Passe - X - 2\pique\ - X : Appel
		
		1\pique\ - Passe - Passe - X - 2\trefle\ - X : Punitif
		
		1\pique\ - Passe - Passe - 2\coeur\ - 2\pique\ - X : Appel
		
		1\pique\ - Passe - Passe - 2\carreau\ - 2\coeur\ - X : Punitif\\
		
		1\pique\ - Passe - 1SA - 2\coeur\ - 2\pique\ - X : Appel
		
		1\pique\ - Passe - 1SA - 2\coeur\ - 3\carreau\ - X : Punitif
		
		1\carreau\ - Passe - 1\coeur\ - X - 2\carreau\ - X : Punitif
		
		1\carreau\ - Passe - 1SA - X - 2\carreau\ - X : Appel\\
		
		1\coeur\ - Passe - 2\coeur\ - 3\carreau\ - 3\coeur\ - X : Appel\\
		
		\SSection{Autres}
		1\coeur\ - Passe - 1\pique\ - Passe - 2\coeur\ - X : Appel
		
		1\coeur\ - Passe - 1SA - Passe - 2\coeur\ - X : Punitif
		
		1\coeur\ - Passe - 1SA - Passe - 2\trefle\ - Passe - 2\coeur\ - Passe - Passe - X : Punitif\\
		
		1\carreau\ - Passe - 1\coeur\ - 2\coeur\ - X : Appel
		
		1\carreau\ - Passe - 1\coeur\ - 2\coeur\ - Passe - Passe - X : Appel\\
		
	\Section{Après 1SA}
		1SA - MULTI - 2\coeur\ - X : Punitif (car couleur devinée)
		
		1SA - LANDY - 2\carreau\ - X : Appel
		
		1SA - 2\coeur\ - 2\pique\ - X : Punitif
		
		1SA - minMAJ - 2\trefle\ - X : Appel (Si 2\trefle\ est naturel, cela devient Passe ou corrige)
		
		1SA - Passe - 2\carreau\ - Passe - 2\coeur\ - 2\pique - X : Appel\\
		
		
		
		1\coeur\ - 1SA - Passe - Passe - X : Du jeu et 6\coeur/4\pique
		
		1\trefle\ - 1SA - Passe - Passe - X : Appel court à \carreau\ (relais à 2\carreau\ nomme ta majeur)
\end{document}